\chapter{Stability Analysis}\label{chapter:stability_analysis}
In \myref[Chapter]{chapter:identify_equilibria}, we have computed all possible equilibria in \myref[Model]{model:3.2}. In this section, we will use mathematical analysis to analyze the stability of each equilibria and determine the conditions for stability. In order to find the stability of each equilibrium point~\cite{book:2478639}, we will need the Jacobian matrix of \myref[Model]{model:3.1}, which is:
\begin{equation}
    \textbf{J}=\begin{bmatrix}
        j_{11} & j_{12} & j_{13}\\
        j_{21} & j_{22} & j_{23}\\
        0 & j_{32} & j_{33}
    \end{bmatrix}
    \label{eq:5.1}
\end{equation}
where
\begin{align*}
    j_{11} &= 1-2x+\gamma_{12}y^2-\gamma_{13}z\\
    j_{12} &= 2\gamma_{12}xy\\
    j_{13} &= -\gamma_{13}x\\
    j_{21} &= 2\gamma_{21}r_1xy\\
    j_{22} &= -\frac{v_1\left(1-p\right)z}{\left(v_1+\left(1-p\right)y\right)^2}+r_1\left(1-2y+\gamma_{21}x^2\right)\\
    j_{23} &= -\frac{\left(1-p\right)y}{v_1+\left(1-p\right)y}\\
    j_{32} &= \frac{v_1v_3\left(1-p\right)z}{\left(v_1+\left(1-p\right)y\right)^2}\\
    j_{33} &= r_2\left(1-2\gamma_{31}z\right)+\frac{v_3\left(1-p\right)y}{v_1+\left(1-p\right)y}-v_2
\end{align*}

\section{Analyzing the trivial equilibrium}\label{sec:stability_trivial_equilibrium}
Plugging the trivial equilibrium into \myref[Matrix]{eq:5.1} yields:
\begin{equation}
    \textbf{J}\left(E_0\right)=\begin{bmatrix}
        1 & 0 & 0\\
        0 & r_1 & 0\\
        0 & 0 & r_2-v_2
    \end{bmatrix}
    \label{eq:5.2}
\end{equation}
The characteristic equation for this Jacobian matrix is:
\begin{equation}
    -\left(\lambda-1\right)\left(\lambda-r_1\right)\left(\lambda-\left(r_2-v_2\right)\right)=0
    \label{eq:5.3}
\end{equation}
Solving for the eigenvalues in \myref[Equation]{eq:5.3}, we get:
\[
\lambda=\left\{
1,\
r_1,\
r_2-v_2
\right\}
\]
Since we have a positive eigenvalue $\lambda=1$, we can conclude that the trivial equilibrium is unstable.

\section{Analyzing the $x$-axial equilibrium}\label{sec:stability_x_axial_equilibrium}
Plugging the $x$-axial equilibrium into \myref[Matrix]{eq:5.1} yields:
\begin{equation}
    \textbf{J}\left(E_x\right)=\begin{bmatrix}
        -1 & 0 & -\gamma_{13}\\
        0 & r_1\left(\gamma_{12}+1\right) & 0\\
        0 & 0 & r_2-v_2
    \end{bmatrix}
    \label{eq:5.4}
\end{equation}
The characteristic equation for this Jacobian matrix is:
\begin{equation}
    -\left(\lambda+1\right)\left(\lambda-r_1\left(\gamma_{12}+1\right)\right)\left(\lambda-\left(r_2-v_2\right)\right)=0
    \label{eq:5.5}
\end{equation}
Solving for the eigenvalues in \myref[Equation]{eq:5.5}, we get:
\[
\lambda=\left\{
-1, \
r_2-v_2, \
r_1\left(\gamma_{12}+1\right)
\right\}
\]
Since the eigenvalue $\lambda=r_1\left(\gamma_{12}+1\right)$ is always positive, we can conclude that the $x$-axial equilibrium is unstable.

\section{Analyzing the $y$-axial equilibrium}\label{sec:stability_y_axial_equilibrium}
Plugging the $y$-axial equilibrium into \myref[Matrix]{eq:5.1} yields:
\begin{equation}
    \textbf{J}\left(E_y\right)=\begin{bmatrix}
        1+\gamma_{12} & 0 & 0\\
        0 & -r_1 & j_{23}\\
        0 & 0 & j_{33}
    \end{bmatrix}
    \label{eq:5.6}
\end{equation}
where
\begin{align*}
    j_{23} &= -\frac{1-p}{v_1+1-p}\\
    j_{33} &= -v_3j_{23}+r_2-v_2
\end{align*}
The characteristic equation for this Jacobian matrix is:
\begin{equation}
    -\left(\lambda-\left(1+\gamma_{12}\right)\right)\left(\lambda+r_1\right)\left(\lambda-j_{33}\right)=0
    \label{eq:5.7}
\end{equation}
Solving for the eigenvalues in \myref[Equation]{eq:5.7}, we get:
\[
\lambda=\left\{
1+\gamma_{12}, \
-r_1, \
-v_3j_{23}+r_2-v_2
\right\}
\]
Since the eigenvalue $\lambda=1+\gamma_{12}$ is always positive, we can conclude that the $y$-axial equilibrium is unstable.

\section{Analyzing the $z$-axial equilibrium}\label{sec:stability_z_axial_equilibrium}
Plugging the $z$-axial equilibrium into \myref[Matrix]{eq:5.1} yields:
\begin{equation}
    \textbf{J}\left(E_z\right)=\begin{bmatrix}
        j_{11} & 0 & 0\\
        0 & j_{22} & 0\\
        0 & j_{32} & v_2-r_2
    \end{bmatrix}
    \label{eq:5.8}
\end{equation}
where
\begin{align*}
    j_{11} &= \frac{\gamma_{13}\left(v_2-r_2\right)+\gamma_{31}r_2}{\gamma_{31}r_2}\\
    j_{22} &= r_1+\frac{\left(1-p\right)\left(v_2-r_2\right)}{\gamma_{31}r_2v_1}\\
    j_{32} &= \frac{v_3\left(r_2-v_2\right)\left(1-p\right)}{\gamma_{31}r_2v_1}
\end{align*}
The characteristic equation for this Jacobian matrix is:
\begin{equation}
    -\left(\lambda-j_{11}\right)\left(\lambda-j_{22}\right)\left(\lambda-j_{33}\right)=0
    \label{eq:5.9}
\end{equation}
Solving for the eigenvalues in \myref[Equation]{eq:5.9}, we get:
\[
\lambda=\left\{
j_{11}, \
j_{22}, \
j_{33}
\right\}
\]
The first eigenvalue is negative when:
\[
\frac{\gamma_{31}r_2}{\gamma_{13}}<r_2-v_2
\]
The second eigenvalue is negative when:
\[
\frac{\gamma_{31}r_1r_2v_1}{1-p}<r_2-v_2
\]
The third eigenvalue is negative when $v_2<r_2$. Therefore we can say that the $z$-axial equilibrium is stable if:
\[
\frac{\gamma_{31}r_2}{\gamma_{13}}<r_2-v_2,\ \frac{\gamma_{31}r_1r_2v_1}{1-p}<r_2-v_2,\ 0<r_2-v_2
\]

\section{Analyzing the $xy$-boundary equilibrium}\label{sec:stability_xy_boundary_equilibrium}
Plugging the $xy$-boundary equilibrium into \myref[Matrix]{eq:5.1} yields:
\begin{equation}
    \textbf{J}\left(E_{xy}\right)=\begin{bmatrix}
        j_{11} & j_{12} & j_{13}\\
        j_{21} & j_{22} & j_{23}\\
        0 & 0 & j_{33}
    \end{bmatrix}
    \label{eq:5.10}
\end{equation}
where
\begin{align*}
    j_{11} &= 1-2\hat{x}+\gamma_{12}\left(\hat{y}\right)^2\\
    j_{12} &= 2\gamma_{12}\hat{x}\hat{y}\\
    j_{13} &= -\gamma_{13}\hat{x}\\
    j_{21} &= 2\gamma_{21}r_1\hat{x}\hat{y}\\
    j_{22} &= r_1\left(1-2\hat{y}+\gamma_{21}\left(\hat{x}\right)^2\right)\\
    j_{23} &= -\frac{\left(1-p\right)\hat{y}}{v_1+\left(1-p\right)\hat{y}}\\
    j_{33} &= r_2+\frac{v_3\left(1-p\right)\hat{y}}{v_1+\left(1-p\right)\hat{y}}-v_2
\end{align*}
The characteristic equation for this Jacobian matrix is:
\begin{equation}
    -\left(j_{33}-\lambda\right)\left(\lambda^2-\left(j_{11}+j_{22}\right)\lambda+j_{11}j_{22}-j_{12}j_{21}\right)=0
    \label{eq:5.11}
\end{equation}
Solving for the eigenvalues in \myref[Equation]{eq:5.11}, we get:
\[
\lambda=\left\{
j_{33},\ 
\frac{\left(j_{11}+j_{22}\right) \pm \sqrt{\left(j_{11}-j_{22}\right)^2+4j_{12}j_{21}}}{2}
\right\}
\]
The first eigenvalue is negative when:
\[
\frac{v_3\left(1-p\right)\hat{y}}{v_1+\left(1-p\right)\hat{y}}<v_2-r_2
\]
The other two eigenvalues are negative when:
\[
4\gamma_{12}\gamma_{21}\left(\hat{x}\right)^2\left(\hat{y}\right)^2<\left(1-2\hat{x}+\gamma_{12}\left(\hat{y}\right)^2\right)\left(1-2\hat{y}+\gamma_{21}\left(\hat{x}\right)^2\right)
\]
Therefore we can say that the $xy$-boundary equilibrium is stable if:
\[
\frac{v_3\left(1-p\right)\hat{y}}{v_1+\left(1-p\right)\hat{y}}<v_2-r_2
\]
and
\[
4\gamma_{12}\gamma_{21}\left(\hat{x}\right)^2\left(\hat{y}\right)^2<\left(1-2\hat{x}+\gamma_{12}\left(\hat{y}\right)^2\right)\left(1-2\hat{y}+\gamma_{21}\left(\hat{x}\right)^2\right)
\]

\section{Analyzing the $xz$-boundary equilibrium}\label{sec:stability_xz_boundary_equilibrium}
Plugging the $xz$-boundary equilibrium into \myref[Matrix]{eq:5.1} yields:
\begin{equation}
    \textbf{J}\left(E_{xz}\right)=\begin{bmatrix}
        j_{11} & 0 & j_{13}\\
        0 & j_{22} & 0\\
        0 & j_{32} & j_{33}
    \end{bmatrix}
    \label{eq:5.12}
\end{equation}
where
\begin{align*}
    j_{11} &= \frac{\gamma_{13}\left(r_2-v_2\right)-\gamma_{31}r_2}{\gamma_{31}r_2}\\
    j_{13} &= \frac{\gamma_{13}\left(\gamma_{13}\left(r_2-v_2\right)-\gamma_{31}r_2\right)}{\gamma_{31}r_2}\\
    j_{22} &= \frac{\gamma_{21}r_1\left(\gamma_{13}\left(r_2-v_2\right)-\gamma_{31}r_2\right)^2}{\gamma_{31}^2r_2^2}+r_1-\frac{\left(r_2-v_2\right)\left(1-p\right)}{\gamma_{31}r_2v_1}\\
    j_{32} &= \frac{v_3\left(r_2-v_2\right)\left(1-p\right)}{\gamma_{31}r_2v_1}\\
    j_{33} &= v_2-r_2
\end{align*}
The characteristic equation for this Jacobian matrix is:
\begin{equation}
    -\left(j_{11}-\lambda\right)\left(j_{22}-\lambda\right)\left(j_{33}-\lambda\right)=0
    \label{eq:5.13}
\end{equation}
Solving for the eigenvalues in \myref[Equation]{eq:5.13}, we get:
\[
\lambda=\left\{
j_{11},\ 
j_{22},\ 
j_{33}
\right\}
\]
The first eigenvalue is negative when:
\[
r_2-v_2<\frac{\gamma_{31}r_2}{\gamma_{13}}
\]
The second eigenvalue is negative when:
\[
\frac{r_1v_1\left(\gamma_{21}\left(\gamma_{13}\left(r_2-v_2\right)-\gamma_{31}r_2\right)^2+\gamma_{31}^2r_2^2\right)}{\gamma_{31}r_2\left(1-p\right)}<r_2-v_2
\]
The third eigenvalue is negative when:
\[
r_2-v_2>0
\]
Therefore we can say that the $xz$-boundary equilibrium is stable if $0<r_2-v_2$, $\displaystyle r_2-v_2<\frac{\gamma_{31}r_2}{\gamma_{13}}$, and:
\[
\frac{r_1v_1\left(\gamma_{21}\left(\gamma_{13}\left(r_2-v_2\right)-\gamma_{31}r_2\right)^2+\gamma_{31}^2r_2^2\right)}{\gamma_{31}r_2\left(1-p\right)}<r_2-v_2
\]

\section{Analyzing the $yz$-boundary equilibrium}\label{sec:stability_yz_boundary_equilibrium}
Plugging the $yz$-boundary equilibrium into \myref[Matrix]{eq:5.1} yields:
\begin{equation}
    \textbf{J}=\begin{bmatrix}
        j_{11} & 0 & 0\\
        0 & j_{22} & j_{23}\\
        0 & j_{32} & j_{33}
    \end{bmatrix}
    \label{eq:5.14}
\end{equation}
where
\begin{align*}
    j_{11} &= 1+\gamma_{12}\left(\bar{y}\right)^2-\gamma_{13}\bar{z}\\
    j_{22} &= r_1\left(1-2\bar{y}\right)-\frac{v_1\left(1-p\right)\bar{z}}{\left(v_1+\left(1-p\right)\bar{y}\right)^2}\\
    j_{23} &= -\frac{\left(1-p\right)\bar{y}}{v_1+\left(1-p\right)\bar{y}}\\
    j_{32} &= \frac{v_1v_3\left(1-p\right)\bar{z}}{\left(v_1+\left(1-p\right)\bar{y}\right)^2}\\
    j_{33} &= r_2\left(1-2\gamma_{31}\bar{z}\right)+\frac{v_3\left(1-p\right)\bar{y}}{v_1+\left(1-p\right)\bar{y}}-v_2
\end{align*}
The characteristic equation for this Jacobian matrix is:
\begin{equation}
    -\left(j_{11}-\lambda\right)\left(\lambda^2-\left(j_{22}+j_{33}\right)\lambda+j_{22}j_{33}-j_{23}j_{32}\right)=0
    \label{eq:5.15}
\end{equation}
Solving for the eigenvalues in \myref[Equation]{eq:5.15}, we get:
\[
\lambda=\left\{
j_{11},\ 
\frac{\left(j_{22}+j_{33}\right)\pm\sqrt{\left(j_{22}-j_{33}\right)^2+4j_{23}j_{32}}}{2}
\right\}
\]
The first eigenvalue is negative when:
\[
\frac{1+\gamma_{12}\bar{y}}{\gamma_{13}}<\bar{z}
\]
The other two eigenvalues are negative when:
\[
j_{23}j_{32}<j_{22}j_{33}
\]
Therefore we can say that the $xy$-boundary equilibrium is stable if:
\[
\frac{1+\gamma_{12}\bar{y}}{\gamma_{13}}<\bar{z},\quad 
j_{23}j_{32}<j_{22}j_{33}
\]
where
\begin{align*}
    j_{22} &= -\frac{v_1\left(1-p\right)\bar{z}}{\left(v_1+\left(1-p\right)\bar{y}\right)^2}+r_1\left(1-2\bar{y}\right)\\
    j_{23} &= -\frac{\left(1-p\right)\bar{y}}{v_1+\left(1-p\right)\bar{y}}\\
    j_{32} &= \frac{v_1v_3\left(1-p\right)\bar{z}}{\left(v_1+\left(1-p\right)\bar{y}\right)^2}\\
    j_{33} &= r_2\left(1-2\gamma_{31}\bar{z}\right)+\frac{v_3\left(1-p\right)\bar{y}}{v_1+\left(1-p\right)\bar{y}}-v_2
\end{align*}

\section{Analyzing the interior equilibrium}\label{sec:stability_interior_equilibrium}
Plugging the interior equilibrium into \myref[Matrix]{eq:5.1} yields:
\begin{equation}
    \textbf{J}=\begin{bmatrix}
        j_{11} & j_{12} & j_{13}\\
        j_{21} & j_{22} & j_{23}\\
        0 & j_{32} & j_{33}
    \end{bmatrix}
    \label{eq:5.16}
\end{equation}
where
\begin{align*}
    j_{11} &= 1-2x^*+\gamma_{12}\left(y^*\right)^2-\gamma_{13}z^*\\
    j_{12} &= 2\gamma_{12}x^*y^*\\
    j_{13} &= -\gamma_{13}x^*\\
    j_{21} &= 2\gamma_{21}r_1x^*y^*\\
    j_{22} &= r_1\left(1-2y^*+\gamma_{21}\left(x^*\right)^2\right)-\frac{v_1\left(1-p\right)z^*}{\left(v_1+\left(1-p\right)y^*\right)^2}\\
    j_{23} &= -\frac{\left(1-p\right)y^*}{v_1+\left(1-p\right)y^*}\\
    j_{32} &= \frac{v_1v_3\left(1-p\right)z^*}{\left(v_1+\left(1-p\right)y^*\right)^2}\\
    j_{33} &= r_2\left(1-2\gamma_{31}z^*\right)+\frac{v_3\left(1-p\right)y^*}{v_1+\left(1-p\right)y^*}-v_2
\end{align*}
The characteristic equation for this Jacobian matrix is:
\begin{equation}
    \lambda^3+J_2\lambda^2+J_1\lambda+J_0=0
    \label{eq:5.17}
\end{equation}
where
\begin{align*}
    J_2 &= -\left(j_{11}+j_{22}+j_{33}\right)\\
    J_1 &= j_{11}\left(j_{22}+j_{33}\right)+j_{22}j_{33}-\left(j_{12}j_{21}+j_{23}j_{32}\right)\\
    J_0 &= -j_{13}j_{21}j_{32}+j_{12}j_{21}j_{33}+j_{11}j_{23}j_{32}-j_{11}j_{22}j_{33}
\end{align*}
By the Routh–Hurwitz stability criterion~\cite{routh1877treatise}, we can say that the interior equilibrium is stable if:
\[
J_2>0,\quad J_1>0,\quad J_0>0,\quad J_2J_1>J_0
\]
where
\begin{align*}
    J_2 &= -\left(j_{11}+j_{22}+j_{33}\right)\\
    J_1 &= j_{11}\left(j_{22}+j_{33}\right)+j_{22}j_{33}-\left(j_{12}j_{21}+j_{23}j_{32}\right)\\
    J_0 &= -j_{13}j_{21}j_{32}+j_{12}j_{21}j_{33}+j_{11}j_{23}j_{32}-j_{11}j_{22}j_{33}\\
    j_{11} &= 1-2x^*+\gamma_{12}\left(y^*\right)^2-\gamma_{13}z^*\\
    j_{12} &= 2\gamma_{12}x^*y^*\\
    j_{13} &= -\gamma_{13}x^*\\
    j_{21} &= 2\gamma_{21}r_1x^*y^*\\
    j_{22} &= r_1\left(1-2y^*+\gamma_{21}\left(x^*\right)^2\right)-\frac{v_1\left(1-p\right)z^*}{\left(v_1+\left(1-p\right)y^*\right)^2}\\
    j_{23} &= -\frac{\left(1-p\right)y^*}{v_1+\left(1-p\right)y^*}\\
    j_{32} &= \frac{v_1v_3\left(1-p\right)z^*}{\left(v_1+\left(1-p\right)y^*\right)^2}\\
    j_{33} &= r_2\left(1-2\gamma_{31}z^*\right)+\frac{v_3\left(1-p\right)y^*}{v_1+\left(1-p\right)y^*}-v_2
\end{align*}