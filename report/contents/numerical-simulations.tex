% ----------------------------------------------------------------------------
% Author: Ramsey (Rayla) Phuc
% Alias: Rayla Kurosaki
% GitHub: https://github.com/rkp1503
% 
% Co-author: Ephraim Agyingi
% ----------------------------------------------------------------------------
\chapter{Numerical Simulations}\label{chapter:numerical_simulations}
In \myref[Chapter]{chapter:stability_analysis}, we used mathematical analysis to analyze the stability of each equilibria and determine the conditions for stability. In this section, we will support and verify the stable equilibria determined in \myref[Chapter]{chapter:stability_analysis} through numerical simulations and compare the interior equilibrium of both models under same/similar parameters.

\section{The $z$-axial equilibrium}\label{sec:numsim_z_axial_equilibrium}
In \myref[Chapter]{sec:z_axial_equilibrium} and \myref[Chapter]{sec:stability_z_axial_equilibrium}, we determined that the $z$-axial equilibrium
\[
E_z=\left(0,0,\frac{r_2-v_2}{\gamma_{31}r_2}\right)
\]
exists if the condition $r_2-v_2>0$ is satisfied and is stable if the following conditions are satisfied:
\[
r_2-v_2>0,\quad 
r_2-v_2>\frac{\gamma_{31}r_2}{\gamma_{13}},\quad 
r_2-v_2>\frac{\gamma_{31}r_1r_2v_1}{1-p}
\]
To satisfy the conditions above, lets consider the following set of parameters:
\begin{equation}
    \begin{dcases}
        \begin{aligned}
            r_1 &= 0.404\\
            r_2 &= 0.903\\
            p &= 0.182
        \end{aligned}
    \end{dcases},\quad 
    \begin{dcases}
        \begin{aligned}
            \gamma_{12} &= 0.639\\
            \gamma_{21} &= 0.283\\
            \gamma_{13} &= 0.301\\
            \gamma_{31} &= 0.110
        \end{aligned}
    \end{dcases},\quad
    \begin{dcases}
        \begin{aligned}
            v_1 &= 0.645\\
            v_2 &= 0.175\\
            v_3 &= 0.145
        \end{aligned}
    \end{dcases} 
    \label{params:6.1}
\end{equation}
Under this set of parameter values, the $z$-axial equilibrium is $E_z=(0,0,1.7343)$. This is further supported by \myref[Figure]{fig:z_axial}, which is the result of numerically solving \myref[Model]{model:3.2}.

\section{The $xy$-boundary equilibrium}\label{sec:numsim_xy_boundary_equilibrium}
In \myref[Chapter]{sec:xy_boundary_equilibrium} and \myref[Chapter]{sec:stability_xy_boundary_equilibrium}, we determined that the $yz$-boundary equilibrium $E_{yz}=\left(\hat{x},\hat{y},0\right)$ exists where
\[
\hat{x}=1+\gamma_{12}\left(\hat{y}\right)^2
\]
and $y^*$ is a positive solution to the equation:
\[
\gamma_{12}^2\gamma_{21}\left(\hat{y}\right)^4+2\gamma_{12}\gamma_{21}\left(\hat{y}\right)^2-\hat{y}+\gamma_{21}+1=0
\]
if the following condition is satisfied for some value of $\beta>0$:
\[
\gamma_{12}<\frac1{\beta^2}\left(\sqrt{\frac{\beta-1}{\gamma_{21}}}-1\right)
\]
and the equilibrium is stable if:
\[
\frac{v_3\left(1-p\right)\hat{y}}{v_1+\left(1-p\right)\hat{y}}<v_2-r_2
\]
and
\[
4\gamma_{12}\gamma_{21}\left(\hat{x}\right)^2\left(\hat{y}\right)^2<\left(1-2\hat{x}+\gamma_{12}\left(\hat{y}\right)^2\right)\left(1-2\hat{y}+\gamma_{21}\left(\hat{x}\right)^2\right)
\]
To satisfy the conditions above, let $\beta=2$ and consider the following set of parameters:
\begin{equation}
    \begin{dcases}
        \begin{aligned}
            r_1 &= 0.978\\
            r_2 &= 0.613\\
            p &= 0.326
        \end{aligned}
    \end{dcases},\quad 
    \begin{dcases}
        \begin{aligned}
            \gamma_{12} &= 0.245\\
            \gamma_{21} &= 0.015\\
            \gamma_{13} &= 0.920\\
            \gamma_{31} &= 0.696
        \end{aligned}
    \end{dcases},\quad
    \begin{dcases}
        \begin{aligned}
            v_1 &= 0.523\\
            v_2 &= 0.951\\
            v_3 &= 0.570
        \end{aligned}
    \end{dcases} 
    \label{params:6.2}
\end{equation}
Under this set of parameter values, the $xy$-boundary equilibrium is $E_{xy}=(1.257,1.024,0)$. This is further supported by \myref[Figure]{fig:xy_boundary}, which is the result of numerically solving \myref[Model]{model:3.2}.

\section{The $xz$-boundary equilibrium}\label{sec:numsim_xz_boundary_equilibrium}
In \myref[Chapter]{sec:xz_boundary_equilibrium} and \myref[Chapter]{sec:stability_xz_boundary_equilibrium}, we determined that the $xz$-boundary equilibrium
\[
E_{xz}=\left(1-\frac{\gamma_{13}\left(r_2-v_2\right)}{\gamma_{31}r_2},0,\frac{r_2-v_2}{\gamma_{31}r_2}\right)
\]
exists if the following conditions are satisfied:
\[
r_2-v_2>0,\quad \frac{\gamma_{31}r_2}{\gamma_{13}}>0
\]
and the equilibrium is stable if $0<r_2-v_2$, $\displaystyle r_2-v_2<\frac{\gamma_{31}r_2}{\gamma_{13}}$, and:
\[
\frac{r_1v_1\left(\gamma_{21}\left(\gamma_{13}\left(r_2-v_2\right)-\gamma_{31}r_2\right)^2+\gamma_{31}^2r_2^2\right)}{\gamma_{31}r_2\left(1-p\right)}<r_2-v_2
\]
To satisfy the conditions above, lets consider the following set of parameters:
\begin{equation}
    \begin{dcases}
        \begin{aligned}
            r_1 &= 0.102\\
            r_2 &= 0.763\\
            p &= 0.271
        \end{aligned}
    \end{dcases},\quad 
    \begin{dcases}
        \begin{aligned}
            \gamma_{12} &= 0.182\\
            \gamma_{21} &= 0.301\\
            \gamma_{13} &= 0.109\\
            \gamma_{31} &= 0.198
        \end{aligned}
    \end{dcases},\quad
    \begin{dcases}
        \begin{aligned}
            v_1 &= 0.983\\
            v_2 &= 0.186\\
            v_3 &= 0.113
        \end{aligned}
    \end{dcases} 
    \label{params:6.3}
\end{equation}
Under this set of parameter values, the $xz$-boundary equilibrium is $E_{xz}=(0.584,0,3.819)$. This is further supported by \myref[Figure]{fig:xz_boundary}, which is the result of numerically solving \myref[Model]{model:3.2}.

\section{The $yz$-boundary equilibrium}\label{sec:numsim_yz_boundary_equilibrium}
In \myref[Chapter]{sec:yz_boundary_equilibrium} and \myref[Chapter]{sec:stability_yz_boundary_equilibrium}, we determined that the $yz$-boundary equilibrium $E_{yz}=\left(0,\bar{y},\bar{z}\right)$ exists where:
\[
\bar{z}=\frac{r_2-v_2}{\gamma_{31}r_2}+\frac{v_3\left(1-p\right)\bar{y}}{\gamma_{31}r_2\left(v_1+\left(1-p\right)\bar{y}\right)}
\]
with a condition that $r_2>v_2$ and $\bar{y}$ is a positive root to the equation:
\begin{equation*}
    Y_3\left(\bar{y}\right)^3+Y_2\left(\bar{y}\right)^2+Y_1\bar{y}+Y_0=0
\end{equation*}
where
\begin{align*}
    Y_3 &= -\gamma_{31}r_1r_2\left(1-p\right)^2\\
    Y_2 &= \gamma_{31}r_1r_2\left(\left(1-p\right)-2v_1\right)\left(1-p\right)\\
    Y_1 &= \gamma_{31}r_1r_2v_1\left(2\left(1-p\right)-v_1\right)+\left(v_2-v_3-r_2\right)\left(1-p\right)^2\\
    Y_0 &= \gamma_{31}r_1r_2v_1^2+v_1\left(v_2-r_2\right)\left(1-p\right)
\end{align*}
with a condition that $1-p>2v_1$ and the equilibrium is stable if:
\[
\frac{1+\gamma_{12}\bar{y}}{\gamma_{13}}<\bar{z},\quad 
j_{23}j_{32}<j_{22}j_{33}
\]
where
\begin{align*}
    j_{22} &= -\frac{v_1\left(1-p\right)\bar{z}}{\left(v_1+\left(1-p\right)\bar{y}\right)^2}+r_1\left(1-2\bar{y}\right)\\
    j_{23} &= -\frac{\left(1-p\right)\bar{y}}{v_1+\left(1-p\right)\bar{y}}\\
    j_{32} &= \frac{v_1v_3\left(1-p\right)\bar{z}}{\left(v_1+\left(1-p\right)\bar{y}\right)^2}\\
    j_{33} &= r_2\left(1-2\gamma_{31}\bar{z}\right)+\frac{v_3\left(1-p\right)\bar{y}}{v_1+\left(1-p\right)\bar{y}}-v_2
\end{align*}
To satisfy the conditions above, lets consider the following set of parameters:
\begin{equation}
    \begin{dcases}
        \begin{aligned}
            r_1 &= 0.978\\
            r_2 &= 0.310\\
            p &= 0.843
        \end{aligned}
    \end{dcases},\quad 
    \begin{dcases}
        \begin{aligned}
            \gamma_{12} &= 0.002\\
            \gamma_{21} &= 0.407\\
            \gamma_{13} &= 0.859\\
            \gamma_{31} &= 0.446
        \end{aligned}
    \end{dcases},\quad
    \begin{dcases}
        \begin{aligned}
            v_1 &= 0.872\\
            v_2 &= 0.201\\
            v_3 &= 0.959
        \end{aligned}
    \end{dcases} 
    \label{params:6.4}
\end{equation}
Under this set of parameter values, the $yz$-boundary equilibrium is $E_{yz}=(0,0.74,1.603)$. This is further supported by \myref[Figure]{fig:yz_boundary}, which is the result of numerically solving \myref[Model]{model:3.2}.

\section{The interior equilibrium}\label{sec:numsim_interior_equilibrium}
In \myref[Chapter]{sec:interior_equilibrium} and \myref[Chapter]{sec:stability_interior_equilibrium}, we determined that the interior equilibrium $E_{xyz}=\left(x^*,y^*,z^*\right)$ exists where:
\[
x^*=1+\gamma_{12}\left(y^*\right)^2-\gamma_{13}z^*
\]
and
\[
z^*=\frac{r_2-v_2}{\gamma_{31}r_2}+\frac{v_3\left(1-p\right)y^*}{\gamma_{31}r_2\left(v_1+\left(1-p\right)y^*\right)};\quad r_2>v_2
\]
and $y^*$ is a positive root to the equation:
\begin{equation*}
    \sum_{i=0}^6 Y_i\left(y^*\right)^i=0
\end{equation*}
where
\begin{align*}
    Y_6 &= \gamma_{12}^2\gamma_{21}\gamma_{31}^2r_1r_2^2\left(1-p\right)^2\\
    Y_5 &= 2\gamma_{12}^2\gamma_{21}\gamma_{31}^2r_1r_2^2v_1\left(1-p\right)\\
    Y_4 &= \gamma_{12}\gamma_{21}\gamma_{31}r_1r_2\left(2\gamma_{13}\left(v_2-r_2-v_3\right)\left(1-p\right)^2+\gamma_{31}r_2\left(\gamma_{12}v_1^2+2\left(1-p\right)^2\right)\right)\\
    Y_3 &= \gamma_{31}r_1r_2\left(2\gamma_{12}\gamma_{13}\gamma_{21}v_1\left(2\left(v_2-r_2\right)-v_3\right)+\gamma_{31}r_2\left(4\gamma_{12}\gamma_{21}v_1-\left(1-p\right)\right)\right)\left(1-p\right)\\
    Y_2 &= r_1\left(2\gamma_{12}\gamma_{13}\gamma_{21}\gamma_{31}r_2v_1^2\left(v_2-r_2\right)+\gamma_{13}^2\gamma_{21}\left(r_2-v_2+v_3\right)^2\left(1-p\right)^2+2\gamma_{13}\gamma_{21}\gamma_{31}r_2\left(v_2-r_2-v_3\right)\left(1-p\right)^2\right.\\
    &\left.+\gamma_{21}\gamma_{31}^2r_2^2\left(2\gamma_{12}v_1^2+\left(1-p\right)^2\right)+\gamma_{31}^2r_2^2\left(\left(1-p\right)-2v_1\right)\left(1-p\right)\right)\\
    Y_1 &= 2\gamma_{13}^2\gamma_{21}r_1v_1\left(r_2-v_2\right)\left(r_2-v_2+v_3\right)\left(1-p\right)+2\gamma_{13}\gamma_{21}\gamma_{31}r_1r_2v_1\left(2\left(v_2-r_2\right)-v_3\right)\left(1-p\right)\\
    &+\gamma_{31}^2r_1r_2^2v_1\left(2\left(\gamma_{21}+1\right)\left(1-p\right)-v_1\right)+\gamma_{31}r_2\left(v_2-r_2-v_3\right)\left(1-p\right)^2\\
    Y_0 &= v_1\left(\gamma_{13}^2\gamma_{21}r_1v_1\left(r_2-v_2\right)^2+\gamma_{31}^2r_1r_2^2v_1\left(\gamma_{21}+1\right)+\gamma_{31}r_2\left(2\gamma_{13}\gamma_{21}r_1v_1+\left(1-p\right)\right)\left(v_2-r_2\right)\right)
\end{align*}
and the equilibrium is stable if:
\[
J_2>0,\quad J_1>0,\quad J_0>0,\quad J_2J_1>J_0
\]
where
\begin{align*}
    J_2 &= -\left(j_{11}+j_{22}+j_{33}\right)\\
    J_1 &= j_{11}\left(j_{22}+j_{33}\right)+j_{22}j_{33}-\left(j_{12}j_{21}+j_{23}j_{32}\right)\\
    J_0 &= -j_{13}j_{21}j_{32}+j_{12}j_{21}j_{33}+j_{11}j_{23}j_{32}-j_{11}j_{22}j_{33}\\
    j_{11} &= 1-2x^*+\gamma_{12}\left(y^*\right)^2-\gamma_{13}z^*\\
    j_{12} &= 2\gamma_{12}x^*y^*\\
    j_{13} &= -\gamma_{13}x^*\\
    j_{21} &= 2\gamma_{21}r_1x^*y^*\\
    j_{22} &= r_1\left(1-2y^*+\gamma_{21}\left(x^*\right)^2\right)-\frac{v_1\left(1-p\right)z^*}{\left(v_1+\left(1-p\right)y^*\right)^2}\\
    j_{23} &= -\frac{\left(1-p\right)y^*}{v_1+\left(1-p\right)y^*}\\
    j_{32} &= \frac{v_1v_3\left(1-p\right)z^*}{\left(v_1+\left(1-p\right)y^*\right)^2}\\
    j_{33} &= r_2\left(1-2\gamma_{31}z^*\right)+\frac{v_3\left(1-p\right)y^*}{v_1+\left(1-p\right)y^*}-v_2
\end{align*}
To ensure that the interior equilibrium exist and is stable, lets consider the following set of parameters:
\begin{equation}
    \begin{dcases}
        \begin{aligned}
            r_1 &= 0.635\\
            r_2 &= 0.742\\
            p &= 0.853
        \end{aligned}
    \end{dcases},\quad 
    \begin{dcases}
        \begin{aligned}
            \gamma_{12} &= 0.142\\
            \gamma_{21} &= 0.002\\
            \gamma_{13} &= 0.148\\
            \gamma_{31} &= 0.215
        \end{aligned}
    \end{dcases},\quad
    \begin{dcases}
        \begin{aligned}
            v_1 &= 0.090\\
            v_2 &= 0.891\\
            v_3 &= 0.980
        \end{aligned}
    \end{dcases} 
    \label{params:6.5}
\end{equation}
Under this set of parameter values, the interior equilibrium is $E_{xyz}=(0.941,0.174,0.426)$. This is further supported by \myref[Figure]{fig:interior}, \myref[Figure]{fig:phase_plane_3d}, \myref[Figure]{fig:phase_plane_xy}, \myref[Figure]{fig:phase_plane_xz}, and \myref[Figure]{fig:phase_plane_yz} where \myref[Figure]{fig:interior} shows the time evolution of each species, \myref[Figure]{fig:phase_plane_3d} shows the phase portrait, and \myref[Figure]{fig:phase_plane_xy}, \myref[Figure]{fig:phase_plane_xz}, and \myref[Figure]{fig:phase_plane_yz} are phase planes when numerically solving \myref[Model]{model:3.2}.
