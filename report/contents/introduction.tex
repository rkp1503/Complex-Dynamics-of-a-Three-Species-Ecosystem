% ----------------------------------------------------------------------------
% Author: Rayla Kurosaki
% GitHub: https://github.com/rkp1503
% ----------------------------------------------------------------------------

\section{Introduction}\label{sec:introduction}
The natural world is a complex web of interactions between organisms, where the survival and prosperity of one species often depend on its relationship with others. In a dynamic ecosystem, like the one under consideration, the intricate dance of mutualism, amensalism, and predation shapes the delicate balance of life. Mutualism, the symbiotic relationship in which two species benefit from each other's presence, plays a pivotal role in maintaining stability and promoting biodiversity in the ecosystem under scrutiny. In contrast, amensalism, an asymmetrical relationship wherein one species is harmed while the other remains unaffected, introduces a contrasting dynamic. Furthermore, the presence of predation, a fundamental aspect of natural selection, adds a layer of complexity to this ecosystem's interplay. The interaction between predator and prey dictates population dynamics, influencing not only the abundance of species but also regulating trophic cascades and maintaining ecological equilibrium. 

Modeling an ecosystem that incorporates mutualism, amensalism, and predation is a multifaceted endeavor, demanding a comprehensive understanding of the intricate relationships and interactions between species. The process involves constructing mathematical equations and simulations to represent the dynamics of these ecological associations. To model mutualism, factors such as resource exchange and fitness benefits must be quantified for the participating species. Incorporating amensalism entails accounting for the interaction effects of competing organisms. Predation modeling necessitates defining predator-prey interactions, predator foraging behavior, and prey population dynamics. By integrating these elements into a cohesive model, we can gain invaluable insights into the stability, resilience, and overall functioning of the ecosystem. Such models are crucial for predicting the consequences of disturbances or perturbations, as well as for formulating effective conservation strategies to safeguard these vital ecological processes.

There have been a plethora of models created to analyze the dynamics of such ecosystems. There are models which considers two species~\cite{GHOSH2017110, CHEN20122790, YU2012208167, HUANG2006672, AZIZALAOUI20031069, XIAO200614, SEN201212, CANTRELL2001206, CHEN20092905, CHEN2010246, KAR2005681, CHATTOPADHYAY1996287, KAR2003125}, three species~\cite{GAKKHAR201654, PANJA2022100153, MENG2014810, ALIDOUSTI2020109688, PEET2005491, SARWARDI2012133, PRIYADARSHI20133202, GAKKHAR2005105, GAKKHAR2007808, Mukherjee2013, DHAKNEMUNDE2012, CHATTOPADHYAY200345, PANJAMONDAL2015, PANJA2017389, KHAJANCHI2017193, JANA2017350} and four species~\cite{JANA2021100942}. Some of these models incorporate a functional response in their model, which include the Beddington–DeAngelis functional response~\cite{CANTRELL2001206}, the Crowley–Martin type functional response~\cite{MENG2014810}, the Holling-type I functional response~\cite{JANA2021100942, Mukherjee2013, CHATTOPADHYAY200345}, the Holling-type II functional response~\cite{GAKKHAR201654, PANJA2022100153, JANA2021100942, GHOSH2017110, YU2012208167, HUANG2006672, AZIZALAOUI20031069, CHEN2010246, SARWARDI2012133, PRIYADARSHI20133202, GAKKHAR2005105, GAKKHAR2007808, Mukherjee2013, CHATTOPADHYAY200345, PANJAMONDAL2015, JANA2017350}, the Holling-type III functional response~\cite{CHATTOPADHYAY200345}, the Leslie-Gower functional response~\cite{YU2012208167, AZIZALAOUI20031069, PRIYADARSHI20133202}, the Monod-Haldane type functional response~\cite{ALIDOUSTI2020109688}, and the Ratio-dependent functional response~\cite{XIAO200614, SEN201212, CANTRELL2001206, KHAJANCHI2017193}. Some models consider prey refuge~\cite{GAKKHAR201654, PANJA2022100153, GHOSH2017110, CHEN20122790, HUANG2006672, CHEN20092905, CHEN2010246, KAR2005681, SARWARDI2012133, Mukherjee2013, KHAJANCHI2017193, JANA2017350}, harvesting~\cite{XIAO200614, KAR2003125, PANJA2017389}, and the Allee effect~\cite{SEN201212}.

In this paper, we will consider a biological system that involves three species with each pairing of species have a unique interaction. In particular, we will study an ecosystem which involves predation, non-linear mutualism, and amensalism. The pairing of species that are in a predation interaction incorporates the Holling type II functional response and refuge into consideration. An example of the ecosystem under consideration is the relationship between the drongo (a bird), the meerkat (a small mongoose) found in southern Africa, and predators (such as jackals). The drongo and the meerkat are mostly in a mutualistic relationship, where the bird helps the mammal by giving a warning cry whenever a predator is near. The meerkat often drops its food when running into its burrow for refuge to avoid the predator and the drongo swoops down to get the food, a win-win for all. In this example, the relationship between the predator and the drongo is amensalism in that the bird is not its source of food. The overpresence of the predator will keep scaring the meerkats into hiding and thus less time for foraging which in turn negatively affects the drongo.
