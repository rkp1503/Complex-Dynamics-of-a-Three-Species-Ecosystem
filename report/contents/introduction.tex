% ----------------------------------------------------------------------------
% Author: Rayla Kurosaki
% GitHub: https://github.com/rkp1503
% ----------------------------------------------------------------------------

\chapter{Introduction}\label{ch:introduction}
In the vast realm of biological sciences, understanding the complex interplay of various phenomena is a formidable challenge. Nature's intricacies often defy direct observation and experimentation, necessitating the development of powerful tools that can unravel its hidden patterns. Enter mathematical models, a transformative approach that harnesses the language of mathematics to dissect, analyze, and predict the behavior of biological systems. These models serve as indispensable bridges between theoretical abstractions and empirical realities, enabling scientists to gain deeper insights into the fundamental principles that govern living organisms. By quantifying and formalizing biological processes, mathematical models offer a systematic framework to study intricate dynamics, investigate the consequences of different hypotheses, and guide experimental design. %This paper delves into the multifaceted world of mathematical modeling in biology, shedding light on its significance, challenges, and transformative potential in illuminating the mysteries of life itself.

There have been a plethora of models created to analyze the dymanics of such ecosystems. There are models which considers two species \cite{GHOSH2017110, CHEN20122790, YU2012208167, HUANG2006672, AZIZALAOUI20031069, XIAO200614, SEN201212, CANTRELL2001206, CHEN20092905, CHEN2010246, KAR2005681, CHATTOPADHYAY1996287, KAR2003125}, three species \cite{GAKKHAR201654, PANJA2022100153, MENG2014810, ALIDOUSTI2020109688, PEET2005491, SARWARDI2012133, PRIYADARSHI20133202, GAKKHAR2005105, GAKKHAR2007808, Mukherjee2013, DHAKNEMUNDE2012, CHATTOPADHYAY200345, PANJAMONDAL2015, PANJA2017389, KHAJANCHI2017193, JANA2017350} and four species \cite{JANA2021100942}. Some of these models incoprorate a functional response in their model, which include the Beddington–DeAngelis functional response \cite{CANTRELL2001206}, the Crowley–Martin type functional response \cite{MENG2014810}, the Holling-type I functional response \cite{JANA2021100942, Mukherjee2013, CHATTOPADHYAY200345}, the Holling-type II functional response \cite{GAKKHAR201654, PANJA2022100153, JANA2021100942, GHOSH2017110, YU2012208167, HUANG2006672, AZIZALAOUI20031069, CHEN2010246, SARWARDI2012133, PRIYADARSHI20133202, GAKKHAR2005105, GAKKHAR2007808, Mukherjee2013, CHATTOPADHYAY200345, PANJAMONDAL2015, JANA2017350}, the Holling-type III functional response \cite{CHATTOPADHYAY200345}, the Leslie-Gower functional response \cite{YU2012208167, AZIZALAOUI20031069, PRIYADARSHI20133202}, the Monod-Haldane type functional response \cite{ALIDOUSTI2020109688}, and the Ratio-dependent functional response \cite{XIAO200614, SEN201212, CANTRELL2001206, KHAJANCHI2017193}. Some models consider prey refuge \cite{GAKKHAR201654, PANJA2022100153, GHOSH2017110, CHEN20122790, HUANG2006672, CHEN20092905, CHEN2010246, KAR2005681, SARWARDI2012133, Mukherjee2013, KHAJANCHI2017193, JANA2017350}, harvesting \cite{XIAO200614, KAR2003125, PANJA2017389}, and the Allee effect \cite{SEN201212}.

In this paper, we will consider a biological system that involves three species with each pairing of species have an unique interaction. In particular, we will study an ecosystem which involves predation, non-linear mutualism, and amensalism. The pairing of species that are in a predation interaction incorporrates the Holling type II functional response and refuge into consideration.  
