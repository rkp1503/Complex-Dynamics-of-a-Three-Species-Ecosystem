%-----------------------------------------------------------------------------
% Author: Rayla Kurosaki
% GitHub: https://github.com/rkp1503
%-----------------------------------------------------------------------------
\documentclass{rayla-project}
\usepackage{rayla-math-style}

%-----------------------------------------------------------------------------
% Project Information
%-----------------------------------------------------------------------------
\myUniversity{
    Rochester Institute of Technology\\
    College of Science\\
    School of Mathematical Sciences
    }
\myTitle{Complex Dynamics of a Three Species Ecosystem}
\myName{Ramsey (Rayla) Phuc}
\courseid{MATH-761.01}
\courseName{Mathematical Biology}
\professorName{Dr. Ephraim Agyingi}
\term{2022/08/22 - 2022/12/05}
\dueDate{2022/12/12}

%----------------------------------------------------------------------------- 
% Start of Document
%-----------------------------------------------------------------------------
\begin{document}

    \maketitle

    %-------------------------------------------------------------------------
    % Abstract
    %-------------------------------------------------------------------------
    \begin{abstract}
        This paper investigates the dynamics of a model system by modifying key assumptions to explore alternative ecological interactions. The original model, outlined in the paper \cite{PANJA2022100153}, initially assumes a competition interaction between species $X$ and $Y$. However, we consider the implications of assuming a mutualism interaction instead. Additionally, we modify the assumption of a commensalism interaction between species $X$ and $Z$ to examine the consequences of an amensalism interaction. Furthermore, the model is expanded by assuming a logistic growth pattern for species Z. Through these modifications, we determine the new equilibrium points and conduct a comprehensive analysis of their characteristics. Numerical computations are performed to assess the stability or instability of the newly derived equilibrium points. The obtained results offer valuable insights into the dynamics and stability of the modified model, shedding light on the consequences of mutualism and amensalism interactions in the examined ecological system. This research contributes to a deeper understanding of the interrelationships between species interactions and population dynamics in ecological systems.
    \end{abstract}

    %-------------------------------------------------------------------------
    % Body
    %-------------------------------------------------------------------------
    % ----------------------------------------------------------------------------
% Author: Rayla Kurosaki
% GitHub: https://github.com/rkp1503
% ----------------------------------------------------------------------------

\section{Introduction}\label{sec:introduction}
The natural world is a complex web of interactions between organisms, where the survival and prosperity of one species often depend on its relationship with others. In a dynamic ecosystem, like the one under consideration, the intricate dance of mutualism, amensalism, and predation shapes the delicate balance of life. Mutualism, the symbiotic relationship in which two species benefit from each other's presence, plays a pivotal role in maintaining stability and promoting biodiversity in the ecosystem under scrutiny. In contrast, amensalism, an asymmetrical relationship wherein one species is harmed while the other remains unaffected, introduces a contrasting dynamic. Furthermore, the presence of predation, a fundamental aspect of natural selection, adds a layer of complexity to this ecosystem's interplay. The interaction between predator and prey dictates population dynamics, influencing not only the abundance of species but also regulating trophic cascades and maintaining ecological equilibrium. 

Modeling an ecosystem that incorporates mutualism, amensalism, and predation is a multifaceted endeavor, demanding a comprehensive understanding of the intricate relationships and interactions between species. The process involves constructing mathematical equations and simulations to represent the dynamics of these ecological associations. To model mutualism, factors such as resource exchange and fitness benefits must be quantified for the participating species. Incorporating amensalism entails accounting for the interaction effects of competing organisms. Predation modeling necessitates defining predator-prey interactions, predator foraging behavior, and prey population dynamics. By integrating these elements into a cohesive model, we can gain invaluable insights into the stability, resilience, and overall functioning of the ecosystem. Such models are crucial for predicting the consequences of disturbances or perturbations, as well as for formulating effective conservation strategies to safeguard these vital ecological processes.

There have been a plethora of models created to analyze the dynamics of such ecosystems. There are models which considers two species~\cite{GHOSH2017110, CHEN20122790, YU2012208167, HUANG2006672, AZIZALAOUI20031069, XIAO200614, SEN201212, CANTRELL2001206, CHEN20092905, CHEN2010246, KAR2005681, CHATTOPADHYAY1996287, KAR2003125}, three species~\cite{GAKKHAR201654, PANJA2022100153, MENG2014810, ALIDOUSTI2020109688, PEET2005491, SARWARDI2012133, PRIYADARSHI20133202, GAKKHAR2005105, GAKKHAR2007808, Mukherjee2013, DHAKNEMUNDE2012, CHATTOPADHYAY200345, PANJAMONDAL2015, PANJA2017389, KHAJANCHI2017193, JANA2017350} and four species~\cite{JANA2021100942}. Some of these models incorporate a functional response in their model, which include the Beddington–DeAngelis functional response~\cite{CANTRELL2001206}, the Crowley–Martin type functional response~\cite{MENG2014810}, the Holling-type I functional response~\cite{JANA2021100942, Mukherjee2013, CHATTOPADHYAY200345}, the Holling-type II functional response~\cite{GAKKHAR201654, PANJA2022100153, JANA2021100942, GHOSH2017110, YU2012208167, HUANG2006672, AZIZALAOUI20031069, CHEN2010246, SARWARDI2012133, PRIYADARSHI20133202, GAKKHAR2005105, GAKKHAR2007808, Mukherjee2013, CHATTOPADHYAY200345, PANJAMONDAL2015, JANA2017350}, the Holling-type III functional response~\cite{CHATTOPADHYAY200345}, the Leslie-Gower functional response~\cite{YU2012208167, AZIZALAOUI20031069, PRIYADARSHI20133202}, the Monod-Haldane type functional response~\cite{ALIDOUSTI2020109688}, and the Ratio-dependent functional response~\cite{XIAO200614, SEN201212, CANTRELL2001206, KHAJANCHI2017193}. Some models consider prey refuge~\cite{GAKKHAR201654, PANJA2022100153, GHOSH2017110, CHEN20122790, HUANG2006672, CHEN20092905, CHEN2010246, KAR2005681, SARWARDI2012133, Mukherjee2013, KHAJANCHI2017193, JANA2017350}, harvesting~\cite{XIAO200614, KAR2003125, PANJA2017389}, and the Allee effect~\cite{SEN201212}.

In this paper, we will consider a biological system that involves three species with each pairing of species have a unique interaction. In particular, we will study an ecosystem which involves predation, non-linear mutualism, and amensalism. The pairing of species that are in a predation interaction incorporates the Holling type II functional response and refuge into consideration. An example of the ecosystem under consideration is the relationship between the drongo (a bird), the meerkat (a small mongoose) found in southern Africa, and predators (such as jackals). The drongo and the meerkat are mostly in a mutualistic relationship, where the bird helps the mammal by giving a warning cry whenever a predator is near. The meerkat often drops its food when running into its burrow for refuge to avoid the predator and the drongo swoops down to get the food, a win-win for all. In this example, the relationship between the predator and the drongo is amensalism in that the bird is not its source of food. The overpresence of the predator will keep scaring the meerkats into hiding and thus less time for foraging which in turn negatively affects the drongo.

    \section{Original Model}\label{sec:original_model}
In this section, we will go over the model that was featured in the paper written by Panja, Gayen, Kar, and Jana~\cite{PANJA2022100153}. The authors wanted to build a model of three species $X$, $Y$, $Z$ which contains three different types of species interactions. Namely nonlinear competition, predation, and commensalism.

\subsection{Assumptions on species $X$}\label{subsec:assumptions_x}
The authors have assumed that $X$ grows logistically at a rate of $r_x$ with a maximum capacity of $K_x$. The authors assumed that $X$ and $Y$ have a competition interaction with an interspecies competition coefficient $\alpha_{xy}$. The authors have also assumed that $X$ and $Z$ are in a commensalism interaction where $Z$ is a host of $X$ with a commensal coefficient $\delta$. From this, the authors have created the differential equation to represent the population of $X$:
\begin{equation}
    \diff[]{X}{T} = r_xX-\frac{r_x}{K_x}X^2-\frac{r_x\alpha_{xy}}{K_x}XY^2+\delta XZ
    \label{eq:2.1}
\end{equation}

\subsection{Assumptions on species $Y$}\label{subsec:assumptions_y}
The authors have assumed that $Y$ grows logistically at a rate of $r_y$ with a maximum capacity of $K_y$. The authors assumed that $Y$ and $X$ have a competition interaction with an interspecies competition coefficient $\alpha_{yx}$. The authors have also assumed that $Y$ and $Z$ are in a predation interaction where $Z$ preys on $Y$ via the Holling type II functional response with a half saturation constant $b$. $Z$ attacks $Y$ at a rate of $a$. Furthermore, as a result of being the prey to $Z$, $Y$ shows refuge behavior with a refuge rate of $p$. From this, the authors have created the differential equation to represent the population of $Y$:
\begin{equation}
    \diff[]{Y}{T} = r_yY-\frac{r_y}{K_y}Y^2-\frac{r_y\alpha_{yx}}{K_y}X^2Y-\frac{a\left(1-p\right)YZ}{b+\left(1-p\right)Y}
    \label{eq:2.2}
\end{equation}

\subsection{Assumptions on species $Z$}\label{subsec:assumptions_z}
The authors assumed that $Z$ grows as a result of being in a predation interaction with $Y$ via the Holling type II functional response with a half saturation constant $b$ and $c$ as the conservation rate of $Y$. Furthermore, the authors assume that the population of $Z$ decays at a rate of $e$. From this, the authors have created the differential equation to represent the population of $Z$:
\begin{equation}
    \diff[]{Z}{T} = \frac{ac\left(1-p\right)YZ}{b+\left(1-p\right)Y}-eZ
    \label{eq:2.3}
\end{equation}

\subsection{Building the model}\label{subsec:model}
From the assumptions, the governing equations for this model are:
\begin{subequations}
    \begin{align}
        \diff[]{X}{T} &= r_xX-\frac{r_x}{K_x}X^2-\frac{r_x\alpha_{xy}}{K_x}XY^2+\delta XZ
        \label{model:2.1a}\\
        \diff[]{Y}{T} &= r_yY-\frac{r_y}{K_y}Y^2-\frac{r_y\alpha_{yx}}{K_y}X^2Y-\frac{a\left(1-p\right)YZ}{b+\left(1-p\right)Y}
        \label{model:2.b}\\
        \diff[]{Z}{T} &= \frac{ac\left(1-p\right)YZ}{b+\left(1-p\right)Y}-eZ
        \label{model:2.1c}
    \end{align}
    \label{model:2.1}
\end{subequations}
and with the following substitutions:
\begin{gather*}
    T=\frac{t}{r_x},\ X=K_xx,\ Y=K_yy,\ Z=\frac{r_xK_y}{a}z\\
    \alpha_{xy}=\frac{K_x}{K_y^2}\gamma_{12},\ \alpha_{yx}=\frac{K_y}{K_x^2}\gamma_{21},\ \delta=\frac{a}{K_y}\gamma\\
    r=\frac{r_y}{r_x},\ b=K_yv_1,\ e=r_xv_2,\ c=\frac{r_xv_3}{a}\\
\end{gather*}
the authors have simplified and non-dimensionalized \myref[Model]{model:2.1}. This gives us the following model the authors have used throughout their paper:
\begin{subequations}
    \begin{align}
        \diff[]{x}{t} &= x\left(1-x-\gamma_{12}y^2\right)+\gamma xz
        \label{model:2.2a}\\
        \diff[]{y}{t} &= ry\left(1-y-\gamma_{21}x^2\right)-\frac{\left(1-p\right)yz}{v_1+\left(1-p\right)y}
        \label{model:2.2b}\\
        \diff[]{z}{t} &= z\left(\frac{v_3\left(1-p\right)y}{v_1+\left(1-p\right)y}-v_2\right)
        \label{model:2.2c}
    \end{align}
    \label{model:2.2}
\end{subequations}
    \section{Revised/Modified Model}\label{sec:modified_model}
In \myref[Section]{sec:original_model}, we discussed the model the authors used in their paper and the assumptions that were used to construct it. In this section, we will modify some of the initial assumptions that the authors have made when creating their model to obtain a new model that we will analyze throughout this paper.

\subsection{Modification 1: The interaction between species $X$ and $Y$}\label{subsec:modification_1}
The original model assumes that species $X$ and $Y$ are in a competition interaction. Instead of considering an interaction that negatively impacts both species, lets consider an interaction that benefits both species. Thus, we will assume that species $X$ and $Y$ are in a mutualism interaction.

\subsection{Modification 2: The interaction between species $X$ and $Z$}\label{subsec:modification_2}
The original model assumes that species $X$ and $Z$ are in a commensalism interaction where species $X$ is benefiting from species $Z$. Instead of considering an interaction where species $X$ gets positively impacted, lets consider an interaction where species $X$ gets negatively impacted while being in an interaction with species $Z$. Thus, we will assume that species $X$ and $Z$ are in an amensalism interaction.

\subsection{Modification 3: The growth rate of species $Z$}\label{subsec:modification_3}
The original model assumes that species $Z$ grows solely due to being in a predation interaction with species $Y$. Here, we will assume that species $Z$ not only grows due to being in a predation interaction with species $Y$, but also grows logistically with a rate of $r_z$ with a capacity of $K_z$.

\subsection{Building the modified model}\label{subsec:modified_model}
With these modified assumptions, the authors' original model become:
\begin{subequations}
    \begin{align}
        \diff[]{X}{T} &= r_xX-\frac{r_x}{K_x}X^2+\frac{r_x\alpha_{xy}}{K_x}XY^2-\delta XZ
        \label{model:3.1a}\\
        \diff[]{Y}{T} &= r_yY-\frac{r_y}{K_y}Y^2+\frac{r_y\alpha_{yx}}{K_y}X^2Y-\frac{a\left(1-p\right)YZ}{b+\left(1-p\right)Y}
        \label{model:3.1b}\\
        \diff[]{Z}{T} &= r_zZ-\frac{r_z}{K_z}Z^2+\frac{ac\left(1-p\right)YZ}{b+\left(1-p\right)Y}-eZ
        \label{model:3.1c}
    \end{align}
    \label{model:3.1}
\end{subequations}
and with the following substitutions:
\begin{gather*}
    X=K_xx,\ Y=K_yy,\ Z=\frac{r_xK_y}{a}z,\ T=\frac{1}{r_x}t\\
    r_1=\frac{r_y}{r_x},\ r_2=\frac{r_z}{r_x},\ v_1=\frac{b}{K_y},\ v_2=\frac{e}{r_x},\ v_3=\frac{ac}{r_x}\\
    \gamma_{12}=\frac{\alpha_{xy}K_y^2}{K_x},\ \gamma_{21}=\frac{\alpha_{yx}K_x^2}{K_y},\ \gamma_{13}=\frac{\delta K_y}{a},\ \gamma_{31}=\frac{r_xK_y}{aK_z}\\
\end{gather*}
we can simplify and non-dimensionalize \myref[Model]{model:3.1}. This gives us the following model we will work on throughout this paper:
\begin{subequations}
    \begin{align}
        \diff[]{x}{t} &= x\left(1-x+\gamma_{12}y^2\right)-\gamma_{13}xz
        \label{model:3.2a}\\
        \diff[]{y}{t} &= r_1y\left(1-y+\gamma_{21}x^2\right)-\frac{\left(1-p\right)yz}{v_1+\left(1-p\right)y}
        \label{model:3.2b}\\
        \diff[]{z}{t} &= r_2z\left(1-\gamma_{31}z\right)+z\left(\frac{v_3\left(1-p\right)y}{v_1+\left(1-p\right)y}-v_2\right)
        \label{model:3.2c}
    \end{align}
    \label{model:3.2}
\end{subequations}

% \begin{table*}[b]
%     \centering
%     \begin{tabular}{l|l}\hline
%         Variable & Description \\ \hline
%         $X,Y$ & Densities of two logistically growing competing species.\\
%         $Z$ & Density of a species which preys on $Y$ ($Z$ eats $Y$)\\
%         $r_x,r_y$ & Intrinsic growth rate of species $X$ and $Y$ respectively\\
%         $\alpha_{12},\alpha_{21}$ & Inter-species competition between two species $X$ and $Y$ respectively\\
%         $a_x,a_y$ & Attack rate of species $Z$ on species $X$ and $Y$ respectively.\\
%         $p_x,p_y$ & Refuge rate of species $X$ and $Y$ respectively.\\
%         $c_x,c_y$ & Conservation rate of species $X$ and $Y$ respectively.\\
%         $b$ & Half saturation constant for Holling type II function.\\
%         $e$ & Death rate of $Z$ species.\\\hline
%     \end{tabular}
%     \caption{Caption}
%     \label{tab:3.1}
% \end{table*}

% \lipsum[1]

    \section{Identifying Equilibria}\label{sec:identify_equilibria}
In \myref[Section]{sec:modified_model}, we have modified some of the assumptions the authors have made and created \myref[Model]{model:3.2} based on the new/modified assumptions. In this section, we will find all the equilibria present in our model. To find all the equilibria $\left(x^*,y^*,z^*\right)$ of this system, we will set all the equations in \myref[Model]{model:3.1} equal to 0~\cite{book:2478639}.
\begin{subequations}
    \begin{align}
        0 &= x^*\left(1-x^*+\gamma_{12}\left(y^*\right)^2\right)-\gamma_{13}x^*z^*
        \label{eq:4.1a}\\
        0 &= r_1y^*\left(1-y^*+\gamma_{21}\left(x^*\right)^2\right)-\frac{\left(1-p\right)y^*z^*}{v_1+\left(1-p\right)y^*}
        \label{eq:4.1b}\\
        0 &= r_2z^*\left(1-\gamma_{31}z^*\right)+z^*\left(\frac{v_3\left(1-p\right)y^*}{v_1+\left(1-p\right)y^*}-v_2\right)
        \label{eq:4.1c}
    \end{align}
    \label{eq:4.1}
\end{subequations}

\subsection{Case 1: The trivial equilibrium}\label{subsec:trivial_equilibrium}
The trivial equilibrium is where $\left(x^*.y^*.z^*\right)=(0,0,0)$. To verify that the trivial equilibrium exists in \myref[Model]{model:3.2}, we will plug in $x^*=0$, $y^*=0$, and $z^*=0$ into the model and must conclude that each equation reduces to $0=0$. Plugging $x^*=0$, $y^*=0$, and $z^*=0$ into \myref[Model]{model:3.2} yields:
\begin{align*}
    0 &= (0)\left(1-(0)+\gamma_{12}(0)^2\right)-\gamma_{13}(0)(0)\\
    0 &= r_1(0)\left(1-(0)+\gamma_{21}(0)^2\right)-\frac{\left(1-p\right)(0)(0)}{v_1+\left(1-p\right)(0)}\\
    0 &= r_2(0)\left(1-\gamma_{31}(0)\right)+(0)\left(\frac{v_3\left(1-p\right)(0)}{v_1+\left(1-p\right)(0)}-v_2\right)
\end{align*}
By inspection, we can see that each equation reduces to $0=0$. Thus, we can say that the trivial equilibrium exists:
\[
\fbox{$\displaystyle E_0=\left(0,0,0\right)$}
\]

\subsection{Case 2: The $x$-axial equilibrium}\label{subsec:x_axial_equilibrium}
The $x$-axial equilibrium is an equilibrium where $x^*$ is non-zero and the other components of the equilibrium are 0. Since we are talking about population densities, it does not make sense to consider values of $x^*$ where $x^*<0$. Thus, the conditions to impose when finding the $x$-axial equilibrium are $x^*>0$ and all other components are 0. To find the $x$-axial equilibrium of \myref[Model]{model:3.2}, we will plug in $y^*=0$ and $z^*=0$:
\begin{align*}
    0 &= x^*\left(1-x^*+\gamma_{12}(0)^2\right)-\gamma_{13}x^*(0)\\
    0 &= r_1(0)\left(1-(0)+\gamma_{21}\left(x^*\right)^2\right)-\frac{\left(1-p\right)(0)(0)}{v_1+\left(1-p\right)(0)}\\
    0 &= r_2(0)\left(1-\gamma_{31}(0)\right)+(0)\left(\frac{v_3\left(1-p\right)(0)}{v_1+\left(1-p\right)(0)}-v_2\right)
\end{align*}
which reduces to:
\begin{equation}
    0 = x^*\left(1-x^*\right)
    \label{eq:4.2}
\end{equation}
\myref[Equation]{eq:4.2} yields solutions $x^*=\left\{0, 1\right\}$. However, due the condition $x^*>0$ we imposed, the only valid solution to consider is $x^*=1$. With this value of $x^*$, we have concluded that the $x$-axial equilibrium exists:
\[
\fbox{$\displaystyle E_x=\left(
1,
0,
0
\right)$}
\]

\subsection{Case 3: The $y$-axial equilibrium}\label{subsec:y_axial_equilibrium}
The $y$-axial equilibrium is an equilibrium where $y^*$ is non-zero and the other components of the equilibrium are 0. Since we are talking about population densities, it does not make sense to consider values of $y^*$ where $y^*<0$. Thus, the conditions to impose when finding the $y$-axial equilibrium are $y^*>0$ and all other components are 0. To find the $y$-axial equilibrium of \myref[Model]{model:3.2}, we will plug in $x^*=0$ and $z^*=0$:
\begin{align*}
    0 &= (0)\left(1-(0)+\gamma_{12}\left(y^*\right)^2\right)-\gamma_{13}(0)(0)\\
    0 &= r_1y^*\left(1-y^*+\gamma_{21}(0)^20\right)-\frac{\left(1-p\right)y^*(0)}{v_1+\left(1-p\right)y^*}\\
    0 &= r_2(0)\left(1-\gamma_{31}(0)\right)+(0)\left(\frac{v_3\left(1-p\right)y^*}{v_1+\left(1-p\right)y^*}-v_2\right)
\end{align*}
which reduces to:
\begin{equation}
    r_1y^*\left(1-y^*\right)=0
    \label{eq:4.3}
\end{equation}
Since all parameters are positive, \myref[Equation]{eq:4.3} yields solutions $y^*=\left\{0, 1\right\}$. However, due the condition $y^*>0$ we imposed, the only valid solution to consider is $y^*=1$. With this value of $y^*$, we have concluded that the $y$-axial equilibrium exists:
\[
\fbox{$\displaystyle E_y=\left(
0,
1,
0
\right)$}
\]

\subsection{Case 4: The $z$-axial equilibrium}\label{subsec:z_axial_equilibrium}
The $z$-axial equilibrium is an equilibrium where $z^*$ is non-zero and the other components of the equilibrium are 0. Since we are talking about population densities, it does not make sense to consider values of $z^*$ where $z^*<0$. Thus, the conditions to impose when finding the $z$-axial equilibrium are $z^*>0$ and all other components are 0. To find the $z$-axial equilibrium of \myref[Model]{model:3.2}, we will plug in $x^*=0$ and $y^*=0$:
\begin{align*}
    0 &= (0)\left(1-(0)+\gamma_{12}(0)^2\right)-\gamma_{13}(0)z^*\\
    0 &= r_1(0)\left(1-(0)+\gamma_{21}(0)^2\right)-\frac{\left(1-p\right)(0)z^*}{v_1+\left(1-p\right)(0)}\\
    0 &= r_2z^*\left(1-\gamma_{31}z^*\right)+z^*\left(\frac{v_3\left(1-p\right)(0)}{v_1+\left(1-p\right)(0)}-v_2\right)
\end{align*}
which reduces to:
\begin{equation}
    r_2z^*\left(1-\gamma_{31}z^*\right)-v_2z^*=0
    \label{eq:4.4}
\end{equation}
Since all parameters are positive, \myref[Equation]{eq:4.3} yields solutions
\[
z^*=\left\{0, \frac{r_2-v_2}{\gamma_{31}r_2}\right\}
\]
However, due the condition $z^*>0$ we imposed, the only valid solution to consider is:
\[
z^*=\frac{r_2-v_2}{\gamma_{31}r_2}
\]
With this value of $z^*$, we have concluded that the $z$-axial equilibrium exists:
\[
\fbox{$\displaystyle E_z=\left(
0,
0,
\frac{r_2-v_2}{\gamma_{31}r_2}
\right)$}
\]
under the condition:
\[
r_2>v_2
\]

\subsection{Case 5: The $xy$-boundary equilibrium}\label{subsec:xy_boundary_equilibrium}
The $xy$-boundary equilibrium is an equilibrium where $x^*$ and $y^*$ are non-zero and all other components of the equilibrium are 0. Since we are talking about population densities, it does not make sense to consider values of $x^*$ and $y^*$ where $x^*<0$ and $y^*<0$. Thus, the conditions to impose when finding the $xy$-boundary equilibrium are $x^*>0$ and $y^*>0$ and all other components are 0. To find the $xy$-boundary equilibrium of \myref[Model]{model:3.2}, we will plug in $z^*=0$:
\begin{align*}
    0 &= x^*\left(1-x^*+\gamma_{12}\left(y^*\right)^2\right)-\gamma_{13}x^*(0)\\
    0 &= r_1y^*\left(1-y^*+\gamma_{21}\left(x^*\right)^2\right)-\frac{\left(1-p\right)y^*(0)}{v_1+\left(1-p\right)y^*}\\
    0 &= r_2(0)\left(1-\gamma_{31}(0)\right)+(0)\left(\frac{v_3\left(1-p\right)y^*}{v_1+\left(1-p\right)y^*}-v_2\right)
\end{align*}
which reduces to:
\begin{subequations}
    \begin{align}
        0 &= x^*\left(1-x^*+\gamma_{12}\left(y^*\right)^2\right)
        \label{eq:4.5a}\\
        0 &= r_1y^*\left(1-y^*+\gamma_{21}\left(x^*\right)^2\right)
        \label{eq:4.5b}
    \end{align}
    \label{eq:4.5}
\end{subequations}
With the conditions $x^*>0$ and $y^*>0$ we imposed, the \myref[System]{eq:4.5} reduces to:
\begin{subequations}
    \begin{align}
        0 &= 1-x^*+\gamma_{12}\left(y^*\right)^2
        \label{eq:4.6a}\\
        0 &= 1-y^*+\gamma_{21}\left(x^*\right)^2
        \label{eq:4.6b}
    \end{align}
    \label{eq:4.6}
\end{subequations}
We can solve for $x^*$ in \myref[Equation]{eq:4.6a}:
\begin{equation}
    x^*=1+\gamma_{12}\left(y^*\right)^2
    \label{eq:4.7}
\end{equation}
and plug \myref[Equation]{eq:4.7} into \myref[Equation]{eq:4.6b} to get:
\begin{equation}
    \gamma_{12}^2\gamma_{21}\left(y^*\right)^4+2\gamma_{12}\gamma_{21}\left(y^*\right)^2-y^*+\left(\gamma_{21}+1\right)=0
    \label{eq:4.8}
\end{equation}
There isn't a nice closed-form solution to \myref[Equation]{eq:4.8} so we will show that there exist a $y^*$ such that \myref[Equation]{eq:4.8} is satisfied. To do this, we will use the intermediate value theorem~\cite{book:2946356}. Let
\[
f\left(y^*\right)=\gamma_{12}^2\gamma_{21}\left(y^*\right)^4+2\gamma_{12}\gamma_{21}\left(y^*\right)^2-y^*+\left(\gamma_{21}+1\right)
\]
Here, we can see that $f(0)=\gamma_{21}+1$. Since all parameters are positive, this means that $f(0)>0$. Let $f(\beta)<0$ for some value of $\beta$. $f\left(\beta\right)<0$ implies:
\[
\gamma_{12}<\frac1{\beta^2}\left(\sqrt{\frac{\beta-1}{\gamma_{21}}}-1\right)
\]
Therefore we can say that the $xy$-boundary equilibrium $E_{xy}=\left(\hat{x},\hat{y},0\right)$ exists where
\[
\hat{x}=1+\gamma_{12}\left(\hat{y}\right)^2
\]
and $y^*$ is a positive solution to the equation:
\[
\gamma_{12}^2\gamma_{21}\left(\hat{y}\right)^4+2\gamma_{12}\gamma_{21}\left(\hat{y}\right)^2-\hat{y}+\gamma_{21}+1=0
\]
if the following condition is satisfied for some value of $\beta>0$:
\[
\gamma_{12}<\frac1{\beta^2}\left(\sqrt{\frac{\beta-1}{\gamma_{21}}}-1\right)
\]

\subsection{Case 6: The $xz$-boundary equilibrium}\label{subsec:xz_boundary_equilibrium}
The $xz$-boundary equilibrium is an equilibrium where $x^*$ and $z^*$ are non-zero and all other components of the equilibrium are 0. Since we are talking about population densities, it does not make sense to consider values of $x^*$ and $z^*$ where $x^*<0$ and $z^*<0$. Thus, the conditions to impose when finding the $xz$-boundary equilibrium are $x^*>0$ and $z^*>0$ and all other components are 0. To find the $xz$-boundary equilibrium of \myref[Model]{model:3.2}, we will plug in $y^*=0$:
\begin{align*}
    0 &= x^*\left(1-x^*+\gamma_{12}(0)^2\right)-\gamma_{13}x^*z^*\\
    0 &= r_1(0)\left(1-(0)+\gamma_{21}\left(x^*\right)^2\right)-\frac{\left(1-p\right)(0)z^*}{v_1+\left(1-p\right)(0)}\\
    0 &= r_2z^*\left(1-\gamma_{31}z^*\right)+z^*\left(\frac{v_3\left(1-p\right)(0)}{v_1+\left(1-p\right)(0)}-v_2\right)
\end{align*}
which reduces to:
\begin{subequations}
    \begin{align}
        0 &= x^*\left(1-x^*\right)-\gamma_{13}x^*z^*
        \label{eq:4.9a}\\
        0 &= r_2z^*\left(1-\gamma_{31}z^*\right)-v_2z^*
        \label{eq:4.9b}
    \end{align}
    \label{eq:4.9}
\end{subequations}
With the conditions $x^*>0$ and $z^*>0$ we imposed, the \myref[System]{eq:4.9} reduces to:
\begin{subequations}
    \begin{align}
        0 &= 1-x^*-\gamma_{13}z^*
        \label{eq:4.10a}\\
        0 &= r_2\left(1-\gamma_{31}z^*\right)-v_2
        \label{eq:4.10b}
    \end{align}
    \label{eq:4.10}
\end{subequations}
We can solve for $z^*$ in \myref[Equation]{eq:4.10b}:
\begin{equation}
    z^*=\frac{r_2-v_2}{\gamma_{31}r_2}
    \label{eq:4.11}
\end{equation}
To ensure that $z^*>0$, we need to impose the condition $r_2>v_2$. Plugging \myref[Equation]{eq:4.11} into \myref[Equation]{eq:4.10a}, we get:
\begin{equation}
    1-x^*-\gamma_{13}\left(\frac{r_2-v_2}{\gamma_{31}r_2}\right)=0
    \label{eq:4.12}
\end{equation}
Solving for $x^*$ in \myref[Equation]{eq:4.12}, we get:
\begin{equation}
    x^*=1-\frac{\gamma_{13}\left(r_2-v_2\right)}{\gamma_{31}r_2}
    \label{eq:4.13}
\end{equation}
To ensure that $x^*>0$, we need to impose the condition
\[
\frac{\gamma_{31}r_2}{\gamma_{13}}>0
\]
With \myref[Equation]{eq:4.13}, $y^*=0$, and \myref[Equation]{eq:4.11}, we can conclude that the $xz$-boundary equilibrium exists:
\[
\fbox{$\displaystyle E_{xz}=\left(
1-\frac{\gamma_{13}\left(r_2-v_2\right)}{\gamma_{31}r_2},
0,
\frac{r_2-v_2}{\gamma_{31}r_2}
\right)$}
\]
under the conditions:
\[
r_2>v_2,\quad \frac{\gamma_{31}r_2}{\gamma_{13}}>0
\]

\subsection{Case 7: The $yz$-boundary equilibrium}\label{subsec:yz_boundary_equilibrium}
The $yz$-boundary equilibrium is an equilibrium where $y^*$ and $z^*$ are non-zero and all other components of the equilibrium are 0. Since we are talking about population densities, it does not make sense to consider values of $y^*$ and $z^*$ where $y^*<0$ and $z^*<0$. Thus, the conditions to impose when finding the $yz$-boundary equilibrium are $y^*>0$ and $z^*>0$ and all other components are 0. To find the $yz$-boundary equilibrium of \myref[Model]{model:3.2}, we will plug in $x^*=0$:
\begin{align*}
    0 &= (0)\left(1-(0)+\gamma_{12}\left(y^*\right)^2\right)-\gamma_{13}(0)z^*\\
    0 &= r_1y^*\left(1-y^*+\gamma_{21}(0)^2\right)-\frac{\left(1-p\right)y^*z^*}{v_1+\left(1-p\right)y^*}\\
    0 &= r_2z^*\left(1-\gamma_{31}z^*\right)+z^*\left(\frac{v_3\left(1-p\right)y^*}{v_1+\left(1-p\right)y^*}-v_2\right)
\end{align*}
which reduces to:
\begin{subequations}
    \begin{align}
        0 &= r_1y^*\left(1-y^*\right)-\frac{\left(1-p\right)y^*z^*}{v_1+\left(1-p\right)y^*}
        \label{eq:4.14a}\\
        0 &= r_2z^*\left(1-\gamma_{31}z^*\right)+z^*\left(\frac{v_3\left(1-p\right)y^*}{v_1+\left(1-p\right)y^*}-v_2\right)
        \label{eq:4.14b}
    \end{align}
    \label{eq:4.14}
\end{subequations}
With the conditions $y^*>0$ and $z^*>0$ we imposed, the \myref[System]{eq:4.14} reduces to:
\begin{subequations}
    \begin{align}
        0 &= r_1\left(1-y^*\right)-\frac{\left(1-p\right)z^*}{v_1+\left(1-p\right)y^*}
        \label{eq:4.15a}\\
        0 &= r_2\left(1-\gamma_{31}z^*\right)+\left(\frac{v_3\left(1-p\right)y^*}{v_1+\left(1-p\right)y^*}-v_2\right)
        \label{eq:4.15b}
    \end{align}
    \label{eq:4.15}
\end{subequations}
We can solve for $z^*$ in \myref[Equation]{eq:4.15b}:
\begin{equation}
    z^*=\frac{r_2-v_2}{\gamma_{31}r_2}+\frac{v_3\left(1-p\right)y^*}{\gamma_{31}r_2\left(v_1+\left(1-p\right)y^*\right)}
    \label{eq:4.16}
\end{equation}
To ensure that $z^*>0$, we need to impose the condition $r_2>v_2$. To solve for $y^*$, we will plug in \myref[Equation]{eq:4.16} into \myref[Equation]{eq:4.15a} to get the equation:
\begin{equation}
    \frac{Y_3\left(y^*\right)^3+Y_2\left(y^*\right)^2+Y_1y^*+Y_0}{\gamma_{31}r_2\left(v_1+\left(1-p\right)y^*\right)^2}=0
    \label{eq:4.17}
\end{equation}
where
\begin{align*}
    Y_3 &= -\gamma_{31}r_1r_2\left(1-p\right)^2\\
    Y_2 &= \gamma_{31}r_1r_2\left(\left(1-p\right)-2v_1\right)\left(1-p\right)\\
    Y_1 &= \gamma_{31}r_1r_2v_1\left(2\left(1-p\right)-v_1\right)+\left(v_2-v_3-r_2\right)\left(1-p\right)^2\\
    Y_0 &= \gamma_{31}r_1r_2v_1^2+v_1\left(v_2-r_2\right)\left(1-p\right)
\end{align*}
Note that we can eliminate one value of $y^*$ form \myref[Equation]{eq:4.17}:
\[
\left(v_1+\left(1-p\right)y^*\right)^2\neq0 \implies y^*\neq-\frac{v_1}{1-p}
\]
and simplify \myref[Equation]{eq:4.17} to:
\begin{equation}
    Y_3\left(y^*\right)^3+Y_2\left(y^*\right)^2+Y_1y^*+Y_0=0
    \label{eq:4.18}
\end{equation}
With a third degree polynomial with complex coefficients, it will be difficult to derive the closed-form solutions to \myref[Equation]{eq:4.18}. However, we don't need to find the exact form of the $y^*$ component in this equilibrium. It is sufficient to show that a positive solution to \myref[Equation]{eq:4.18} exists. This is because if $y^*=0$, then it will lead to the $z$-axial equilibrium and if $y^*<0$ or if $y^*$ is complex, then we ignore it since it biologically does not make sense. Going through the coefficients of \myref[Equation]{eq:4.18}, we can immediately see that $Y_3<0$. For $Y_2$, we can place a condition to determine when its positive or negative. In particular, we can say that
\[
\begin{dcases}
    Y_2<0 &\text{if}\quad 1-p<2v_1\\
    Y_2>0 &\text{if}\quad 1-p>2v_1
\end{dcases}
\]
For $Y_1$, we have:
\[
\begin{dcases}
    Y_1<0 &\text{if}\quad \frac{\gamma_{31}r_1r_2v_1\left(2\left(1-p\right)-v_1\right)}{\left(r_2-v_2+v_3\right)\left(1-p\right)^2}<1\\
    Y_1>0 &\text{if}\quad \frac{\gamma_{31}r_1r_2v_1\left(2\left(1-p\right)-v_1\right)}{\left(r_2-v_2+v_3\right)\left(1-p\right)^2}>1
\end{dcases}
\]
and for $Y_0$, we have:
\[
\begin{dcases}
    Y_0<0 &\text{if}\quad \frac{\gamma_{31}r_1r_2v_1}{\left(r_2-v_2\right)\left(1-p\right)}<1\\
    Y_0>0 &\text{if}\quad \frac{\gamma_{31}r_1r_2v_1}{\left(r_2-v_2\right)\left(1-p\right)}>1
\end{dcases}
\]
Note that we had to impose a condition on $z^*$, which was $r_2>v_2$. With this condition, we can rewrite one our conditions on $Y_0$:
\[
\begin{dcases}
    Y_0<0 &\text{if}\quad \frac{\gamma_{31}r_1r_2v_1}{1-p}<r_2-v_2\\
    Y_0>0 &\text{if}\quad \frac{\gamma_{31}r_1r_2v_1}{1-p}>0
\end{dcases}
\]
and for $Y_1$, the conditions can be rewritten as:
\[
\begin{dcases}
    Y_1<0 &\text{if}\quad \frac{\gamma_{31}r_1r_2v_1\left(2\left(1-p\right)-v_1\right)}{\left(1-p\right)^2}<r_2-v_2+v_3\\
    Y_1>0 &\text{if}\quad 1-p>\frac{v_1}2
\end{dcases}
\]
By Descartes' rule of signs~\cite{10.2307/1967494}, we can say that \myref[Equation]{eq:4.18} has at least one positive solution if $Y_3<0$ and $Y_2,Y_1,Y_0>0$. This means that we must impose the conditions:
\[
1-p>2v_1,\quad 1-p>\frac{v_1}2,\quad \frac{\gamma_{31}r_1r_2v_1}{1-p}>0
\]
Since all parameters are positive, the third condition is always fulfilled. Also, we know that $2v_1>v_1/2$ so the second condition is redundant. Therefore, we can say that the $yz$-boundary equilibrium $E_{yz}=\left(0,\bar{y},\bar{z}\right)$ exists where:
\[
\bar{z}=\frac{r_2-v_2}{\gamma_{31}r_2}+\frac{v_3\left(1-p\right)\bar{y}}{\gamma_{31}r_2\left(v_1+\left(1-p\right)\bar{y}\right)}
\]
with a condition that $r_2>v_2$ and $\bar{y}$ is a positive root to the equation:
\begin{equation*}
    Y_3\left(\bar{y}\right)^3+Y_2\left(\bar{y}\right)^2+Y_1\bar{y}+Y_0=0
\end{equation*}
where
\begin{align*}
    Y_3 &= -\gamma_{31}r_1r_2\left(1-p\right)^2\\
    Y_2 &= \gamma_{31}r_1r_2\left(\left(1-p\right)-2v_1\right)\left(1-p\right)\\
    Y_1 &= \gamma_{31}r_1r_2v_1\left(2\left(1-p\right)-v_1\right)+\left(v_2-v_3-r_2\right)\left(1-p\right)^2\\
    Y_0 &= \gamma_{31}r_1r_2v_1^2+v_1\left(v_2-r_2\right)\left(1-p\right)
\end{align*}
with a condition that $1-p>2v_1$.

\subsection{Case 8: The interior equilibrium}\label{subsec:interior_equilibrium}
The interior equilibrium is an equilibrium where all the components of the equilibrium are non-zero. Since we are talking about population densities, having non-zero components that are also negative would not make sense. Thus, the conditions to impose when finding the interior equilibrium are when the components are positive. To find the interior equilibrium of \myref[Model]{model:3.2}, we know that $x^*,y^*,z^*\neq0$. So we can reduce the model to:
\begin{subequations}
    \begin{align}
        0 &= 1-x^*+\gamma_{12}\left(y^*\right)^2-\gamma_{13}z^*
        \label{eq:4.21a}\\
        0 &= r_1\left(1-y^*+\gamma_{21}\left(x^*\right)^2\right)-\frac{\left(1-p\right)z^*}{v_1+\left(1-p\right)y^*}
        \label{eq:4.21b}\\
        0 &= r_2\left(1-\gamma_{31}z^*\right)+\left(\frac{v_3\left(1-p\right)y^*}{v_1+\left(1-p\right)y^*}-v_2\right)
        \label{eq:4.21c}
    \end{align}
    \label{eq:4.21}
\end{subequations}
We can solve for $x^*$ in \myref[Equation]{eq:4.21a}:
\begin{equation}
    x^*=1+\gamma_{12}\left(y^*\right)^2-\gamma_{13}z^*
    \label{eq:4.22}
\end{equation}
To ensure that $x^*>0$, we will need to impose the condition:
\[
z^*<\frac{1+\gamma_{12}\left(y^*\right)^2}{\gamma_{13}}
\]
We can also solve for $z^*$ in \myref[Equation]{eq:4.21c}:
\begin{equation}
    z^*=\frac{r_2-v_2}{\gamma_{31}r_2}+\frac{v_3\left(1-p\right)y^*}{\gamma_{31}r_2\left(v_1+\left(1-p\right)y^*\right)}
    \label{eq:4.23}
\end{equation}
To ensure that $z^*>0$, we will need to impose the condition $r_2>v_2$. With \myref[Equation]{eq:4.22} and \myref[Equation]{eq:4.23}, we can plug them into \myref[Equation]{eq:4.21b} to obtain the following Equation:
\begin{equation}
    \frac{1}{\gamma_{13}^2r_2^2\left(v_1+\left(1-p\right)y^*\right)^2}\sum_{i=0}^6 Y_i\left(y^*\right)^i=0
    \label{eq:4.24}
\end{equation}
where
\begin{align*}
    Y_6 &= \gamma_{12}^2\gamma_{21}\gamma_{31}^2r_1r_2^2\left(1-p\right)^2\\
    Y_5 &= 2\gamma_{12}^2\gamma_{21}\gamma_{31}^2r_1r_2^2v_1\left(1-p\right)\\
    Y_4 &= \gamma_{12}\gamma_{21}\gamma_{31}r_1r_2\left(2\gamma_{13}\left(v_2-r_2-v_3\right)\left(1-p\right)^2+\gamma_{31}r_2\left(\gamma_{12}v_1^2+2\left(1-p\right)^2\right)\right)\\
    Y_3 &= \gamma_{31}r_1r_2\left(2\gamma_{12}\gamma_{13}\gamma_{21}v_1\left(2\left(v_2-r_2\right)-v_3\right)+\gamma_{31}r_2\left(4\gamma_{12}\gamma_{21}v_1-\left(1-p\right)\right)\right)\left(1-p\right)\\
    Y_2 &= r_1\left(2\gamma_{12}\gamma_{13}\gamma_{21}\gamma_{31}r_2v_1^2\left(v_2-r_2\right)+\gamma_{13}^2\gamma_{21}\left(r_2-v_2+v_3\right)^2\left(1-p\right)^2+2\gamma_{13}\gamma_{21}\gamma_{31}r_2\left(v_2-r_2-v_3\right)\left(1-p\right)^2\right.\\
    &\left.+\gamma_{21}\gamma_{31}^2r_2^2\left(2\gamma_{12}v_1^2+\left(1-p\right)^2\right)+\gamma_{31}^2r_2^2\left(\left(1-p\right)-2v_1\right)\left(1-p\right)\right)\\
    Y_1 &= 2\gamma_{13}^2\gamma_{21}r_1v_1\left(r_2-v_2\right)\left(r_2-v_2+v_3\right)\left(1-p\right)+2\gamma_{13}\gamma_{21}\gamma_{31}r_1r_2v_1\left(2\left(v_2-r_2\right)-v_3\right)\left(1-p\right)\\
    &+\gamma_{31}^2r_1r_2^2v_1\left(2\left(\gamma_{21}+1\right)\left(1-p\right)-v_1\right)+\gamma_{31}r_2\left(v_2-r_2-v_3\right)\left(1-p\right)^2\\
    Y_0 &= v_1\left(\gamma_{13}^2\gamma_{21}r_1v_1\left(r_2-v_2\right)^2+\gamma_{31}^2r_1r_2^2v_1\left(\gamma_{21}+1\right)+\gamma_{31}r_2\left(2\gamma_{13}\gamma_{21}r_1v_1+\left(1-p\right)\right)\left(v_2-r_2\right)\right)
\end{align*}

Note that we can eliminate one value of $y^*$ form \myref[Equation]{eq:4.24}:
\[
\left(v_1+\left(1-p\right)y^*\right)^2\neq0 \implies y^*\neq-\frac{v_1}{1-p}
\]
and simplify \myref[Equation]{eq:4.24} to:
\begin{equation}
    \sum_{i=0}^6 Y_i\left(y^*\right)^i=0
    \label{eq:4.25}
\end{equation}
What we are left with is an equation in the form of a polynomial of degree 6. The solutions to \myref[Equation]{eq:4.25} cannot be analytically solved. However, we don't need to find the exact form of the $y^*$ component in this equilibrium. It is sufficient to show that a positive solution to \myref[Equation]{eq:4.25} exists. To do this, we will use Descartes' rule of signs. From the coefficients, we can conclude that $Y_5,Y_6>0$ since all parameters are positive. Then, to ensure that \myref[Equation]{eq:4.25} has at least one positive solution, we will need an odd number of sign changes after the fifth degree term. For simplicity, we will make all the other coefficients negative. Thus, we can say that the interior equilibrium $E_{xyz}=\left(x^*,y^*,z^*\right)$ exists where:
\[
x^*=1+\gamma_{12}\left(y^*\right)^2-\gamma_{13}z^*
\]
and
\[
z^*=\frac{r_2-v_2}{\gamma_{31}r_2}+\frac{v_3\left(1-p\right)y^*}{\gamma_{31}r_2\left(v_1+\left(1-p\right)y^*\right)};\quad r_2>v_2
\]
and $y^*$ is a positive root to the equation:
\begin{equation*}
    \sum_{i=0}^6 Y_i\left(y^*\right)^i=0
\end{equation*}
where $Y_0<0$, $Y_1<0$, $Y_2<0$, $Y_3<0$, $Y_4<0$ and 
\begin{align*}
    Y_6 &= \gamma_{12}^2\gamma_{21}\gamma_{31}^2r_1r_2^2\left(1-p\right)^2\\
    Y_5 &= 2\gamma_{12}^2\gamma_{21}\gamma_{31}^2r_1r_2^2v_1\left(1-p\right)\\
    Y_4 &= \gamma_{12}\gamma_{21}\gamma_{31}r_1r_2\left(2\gamma_{13}\left(v_2-r_2-v_3\right)\left(1-p\right)^2+\gamma_{31}r_2\left(\gamma_{12}v_1^2+2\left(1-p\right)^2\right)\right)\\
    Y_3 &= \gamma_{31}r_1r_2\left(2\gamma_{12}\gamma_{13}\gamma_{21}v_1\left(2\left(v_2-r_2\right)-v_3\right)+\gamma_{31}r_2\left(4\gamma_{12}\gamma_{21}v_1-\left(1-p\right)\right)\right)\left(1-p\right)\\
    Y_2 &= r_1\left(2\gamma_{12}\gamma_{13}\gamma_{21}\gamma_{31}r_2v_1^2\left(v_2-r_2\right)+\gamma_{13}^2\gamma_{21}\left(r_2-v_2+v_3\right)^2\left(1-p\right)^2+2\gamma_{13}\gamma_{21}\gamma_{31}r_2\left(v_2-r_2-v_3\right)\left(1-p\right)^2\right.\\
    &\left.+\gamma_{21}\gamma_{31}^2r_2^2\left(2\gamma_{12}v_1^2+\left(1-p\right)^2\right)+\gamma_{31}^2r_2^2\left(\left(1-p\right)-2v_1\right)\left(1-p\right)\right)\\
    Y_1 &= 2\gamma_{13}^2\gamma_{21}r_1v_1\left(r_2-v_2\right)\left(r_2-v_2+v_3\right)\left(1-p\right)+2\gamma_{13}\gamma_{21}\gamma_{31}r_1r_2v_1\left(2\left(v_2-r_2\right)-v_3\right)\left(1-p\right)\\
    &+\gamma_{31}^2r_1r_2^2v_1\left(2\left(\gamma_{21}+1\right)\left(1-p\right)-v_1\right)+\gamma_{31}r_2\left(v_2-r_2-v_3\right)\left(1-p\right)^2\\
    Y_0 &= v_1\left(\gamma_{13}^2\gamma_{21}r_1v_1\left(r_2-v_2\right)^2+\gamma_{31}^2r_1r_2^2v_1\left(\gamma_{21}+1\right)+\gamma_{31}r_2\left(2\gamma_{13}\gamma_{21}r_1v_1+\left(1-p\right)\right)\left(v_2-r_2\right)\right)
\end{align*}

    \section{Stability Analysis via Linearization}\label{sec:stability_analysis}
In \myref[Section]{sec:identify_equilibria}, we have computed all possible equilibria in \myref[Model]{model:3.2}. In this section, we will use mathematical analysis to analyze the stability of each equilibria and determine the conditions for stability. In order to find the stability of each equilibrium point with the linearization method~\cite{book:2478639}, we will need the Jacobian matrix of \myref[Model]{model:3.1}, which is:
\begin{equation}
    \textbf{J}=\begin{bmatrix}
        j_{11} & j_{12} & j_{13}\\
        j_{21} & j_{22} & j_{23}\\
        0 & j_{32} & j_{33}
    \end{bmatrix}
    \label{eq:5.1}
\end{equation}
where
\begin{align*}
    j_{11} &= 1-2x+\gamma_{12}y^2-\gamma_{13}z\\
    j_{12} &= 2\gamma_{12}xy\\
    j_{13} &= -\gamma_{13}x\\
    j_{21} &= 2\gamma_{21}r_1xy\\
    j_{22} &= -\frac{v_1\left(1-p\right)z}{\left(v_1+\left(1-p\right)y\right)^2}+r_1\left(1-2y+\gamma_{21}x^2\right)\\
    j_{23} &= -\frac{\left(1-p\right)y}{v_1+\left(1-p\right)y}\\
    j_{32} &= \frac{v_1v_3\left(1-p\right)z}{\left(v_1+\left(1-p\right)y\right)^2}\\
    j_{33} &= r_2\left(1-2\gamma_{31}z\right)+\frac{v_3\left(1-p\right)y}{v_1+\left(1-p\right)y}-v_2
\end{align*}

\subsection{Analyzing the trivial equilibrium}\label{subsec:stability_trivial_equilibrium}
Plugging the trivial equilibrium into \myref[Matrix]{eq:5.1} yields:
\begin{equation}
    \textbf{J}\left(E_0\right)=\begin{bmatrix}
        1 & 0 & 0\\
        0 & r_1 & 0\\
        0 & 0 & r_2-v_2
    \end{bmatrix}
    \label{eq:5.2}
\end{equation}
The characteristic equation for this Jacobian matrix is:
\begin{equation}
    -\left(\lambda-1\right)\left(\lambda-r_1\right)\left(\lambda-\left(r_2-v_2\right)\right)=0
    \label{eq:5.3}
\end{equation}
Solving for the eigenvalues in \myref[Equation]{eq:5.3}, we get:
\[
\lambda=\left\{
1,\
r_1,\
r_2-v_2
\right\}
\]
Since we have a positive eigenvalue $\lambda=1$, we can conclude that the trivial equilibrium is unstable.

\subsection{Analyzing the $x$-axial equilibrium}\label{subsec:stability_x_axial_equilibrium}
Plugging the $x$-axial equilibrium into \myref[Matrix]{eq:5.1} yields:
\begin{equation}
    \textbf{J}\left(E_x\right)=\begin{bmatrix}
        -1 & 0 & -\gamma_{13}\\
        0 & r_1\left(\gamma_{12}+1\right) & 0\\
        0 & 0 & r_2-v_2
    \end{bmatrix}
    \label{eq:5.4}
\end{equation}
The characteristic equation for this Jacobian matrix is:
\begin{equation}
    -\left(\lambda+1\right)\left(\lambda-r_1\left(\gamma_{12}+1\right)\right)\left(\lambda-\left(r_2-v_2\right)\right)=0
    \label{eq:5.5}
\end{equation}
Solving for the eigenvalues in \myref[Equation]{eq:5.5}, we get:
\[
\lambda=\left\{
-1, \
r_2-v_2, \
r_1\left(\gamma_{12}+1\right)
\right\}
\]
Since the eigenvalue $\lambda=r_1\left(\gamma_{12}+1\right)$ is always positive, we can conclude that the $x$-axial equilibrium is unstable.

\subsection{Analyzing the $y$-axial equilibrium}\label{subsec:stability_y_axial_equilibrium}
Plugging the $y$-axial equilibrium into \myref[Matrix]{eq:5.1} yields:
\begin{equation}
    \textbf{J}\left(E_y\right)=\begin{bmatrix}
        1+\gamma_{12} & 0 & 0\\
        0 & -r_1 & j_{23}\\
        0 & 0 & j_{33}
    \end{bmatrix}
    \label{eq:5.6}
\end{equation}
where
\begin{align*}
    j_{23} &= -\frac{1-p}{v_1+1-p}\\
    j_{33} &= -v_3j_{23}+r_2-v_2
\end{align*}
The characteristic equation for this Jacobian matrix is:
\begin{equation}
    -\left(\lambda-\left(1+\gamma_{12}\right)\right)\left(\lambda+r_1\right)\left(\lambda-j_{33}\right)=0
    \label{eq:5.7}
\end{equation}
Solving for the eigenvalues in \myref[Equation]{eq:5.7}, we get:
\[
\lambda=\left\{
1+\gamma_{12}, \
-r_1, \
-v_3j_{23}+r_2-v_2
\right\}
\]
Since the eigenvalue $\lambda=1+\gamma_{12}$ is always positive, we can conclude that the $y$-axial equilibrium is unstable.

\subsection{Analyzing the $z$-axial equilibrium}\label{subsec:stability_z_axial_equilibrium}
Plugging the $z$-axial equilibrium into \myref[Matrix]{eq:5.1} yields:
\begin{equation}
    \textbf{J}\left(E_z\right)=\begin{bmatrix}
        j_{11} & 0 & 0\\
        0 & j_{22} & 0\\
        0 & j_{32} & v_2-r_2
    \end{bmatrix}
    \label{eq:5.8}
\end{equation}
where
\begin{align*}
    j_{11} &= \frac{\gamma_{13}\left(v_2-r_2\right)+\gamma_{31}r_2}{\gamma_{31}r_2}\\
    j_{22} &= r_1+\frac{\left(1-p\right)\left(v_2-r_2\right)}{\gamma_{31}r_2v_1}\\
    j_{32} &= \frac{v_3\left(r_2-v_2\right)\left(1-p\right)}{\gamma_{31}r_2v_1}
\end{align*}
The characteristic equation for this Jacobian matrix is:
\begin{equation}
    -\left(\lambda-j_{11}\right)\left(\lambda-j_{22}\right)\left(\lambda-j_{33}\right)=0
    \label{eq:5.9}
\end{equation}
Solving for the eigenvalues in \myref[Equation]{eq:5.9}, we get:
\[
\lambda=\left\{
j_{11}, \
j_{22}, \
j_{33}
\right\}
\]
The first eigenvalue is negative when:
\[
\frac{\gamma_{31}r_2}{\gamma_{13}}<r_2-v_2
\]
The second eigenvalue is negative when:
\[
\frac{\gamma_{31}r_1r_2v_1}{1-p}<r_2-v_2
\]
The third eigenvalue is negative when $v_2<r_2$. Therefore we can say that the $z$-axial equilibrium is stable if:
\[
\frac{\gamma_{31}r_2}{\gamma_{13}}<r_2-v_2,\ \frac{\gamma_{31}r_1r_2v_1}{1-p}<r_2-v_2,\ 0<r_2-v_2
\]

\subsection{Analyzing the $xy$-boundary equilibrium}\label{subsec:stability_xy_boundary_equilibrium}
Plugging the $xy$-boundary equilibrium into \myref[Matrix]{eq:5.1} yields:
\begin{equation}
    \textbf{J}\left(E_{xy}\right)=\begin{bmatrix}
        j_{11} & j_{12} & j_{13}\\
        j_{21} & j_{22} & j_{23}\\
        0 & 0 & j_{33}
    \end{bmatrix}
    \label{eq:5.10}
\end{equation}
where
\begin{align*}
    j_{11} &= 1-2\hat{x}+\gamma_{12}\left(\hat{y}\right)^2\\
    j_{12} &= 2\gamma_{12}\hat{x}\hat{y}\\
    j_{13} &= -\gamma_{13}\hat{x}\\
    j_{21} &= 2\gamma_{21}r_1\hat{x}\hat{y}\\
    j_{22} &= r_1\left(1-2\hat{y}+\gamma_{21}\left(\hat{x}\right)^2\right)\\
    j_{23} &= -\frac{\left(1-p\right)\hat{y}}{v_1+\left(1-p\right)\hat{y}}\\
    j_{33} &= r_2+\frac{v_3\left(1-p\right)\hat{y}}{v_1+\left(1-p\right)\hat{y}}-v_2
\end{align*}
The characteristic equation for this Jacobian matrix is:
\begin{equation}
    -\left(j_{33}-\lambda\right)\left(\lambda^2-\left(j_{11}+j_{22}\right)\lambda+j_{11}j_{22}-j_{12}j_{21}\right)=0
    \label{eq:5.11}
\end{equation}
Solving for the eigenvalues in \myref[Equation]{eq:5.11}, we get:
\[
\lambda=\left\{
j_{33},\ 
\frac{\left(j_{11}+j_{22}\right) \pm \sqrt{\left(j_{11}-j_{22}\right)^2+4j_{12}j_{21}}}{2}
\right\}
\]
The first eigenvalue is negative when:
\[
\frac{v_3\left(1-p\right)\hat{y}}{v_1+\left(1-p\right)\hat{y}}<v_2-r_2
\]
The other two eigenvalues are negative when:
\[
4\gamma_{12}\gamma_{21}\left(\hat{x}\right)^2\left(\hat{y}\right)^2<\left(1-2\hat{x}+\gamma_{12}\left(\hat{y}\right)^2\right)\left(1-2\hat{y}+\gamma_{21}\left(\hat{x}\right)^2\right)
\]
Therefore we can say that the $xy$-boundary equilibrium is stable if:
\[
\frac{v_3\left(1-p\right)\hat{y}}{v_1+\left(1-p\right)\hat{y}}<v_2-r_2
\]
and
\[
4\gamma_{12}\gamma_{21}\left(\hat{x}\right)^2\left(\hat{y}\right)^2<\left(1-2\hat{x}+\gamma_{12}\left(\hat{y}\right)^2\right)\left(1-2\hat{y}+\gamma_{21}\left(\hat{x}\right)^2\right)
\]

\subsection{Analyzing the $xz$-boundary equilibrium}\label{subsec:stability_xz_boundary_equilibrium}
Plugging the $xz$-boundary equilibrium into \myref[Matrix]{eq:5.1} yields:
\begin{equation}
    \textbf{J}\left(E_{xz}\right)=\begin{bmatrix}
        j_{11} & 0 & j_{13}\\
        0 & j_{22} & 0\\
        0 & j_{32} & j_{33}
    \end{bmatrix}
    \label{eq:5.12}
\end{equation}
where
\begin{align*}
    j_{11} &= \frac{\gamma_{13}\left(r_2-v_2\right)-\gamma_{31}r_2}{\gamma_{31}r_2}\\
    j_{13} &= \frac{\gamma_{13}\left(\gamma_{13}\left(r_2-v_2\right)-\gamma_{31}r_2\right)}{\gamma_{31}r_2}\\
    j_{22} &= \frac{\gamma_{21}r_1\left(\gamma_{13}\left(r_2-v_2\right)-\gamma_{31}r_2\right)^2}{\gamma_{31}^2r_2^2}+r_1-\frac{\left(r_2-v_2\right)\left(1-p\right)}{\gamma_{31}r_2v_1}\\
    j_{32} &= \frac{v_3\left(r_2-v_2\right)\left(1-p\right)}{\gamma_{31}r_2v_1}\\
    j_{33} &= v_2-r_2
\end{align*}
The characteristic equation for this Jacobian matrix is:
\begin{equation}
    -\left(j_{11}-\lambda\right)\left(j_{22}-\lambda\right)\left(j_{33}-\lambda\right)=0
    \label{eq:5.13}
\end{equation}
Solving for the eigenvalues in \myref[Equation]{eq:5.13}, we get:
\[
\lambda=\left\{
j_{11},\ 
j_{22},\ 
j_{33}
\right\}
\]
The first eigenvalue is negative when:
\[
r_2-v_2<\frac{\gamma_{31}r_2}{\gamma_{13}}
\]
The second eigenvalue is negative when:
\[
\frac{r_1v_1\left(\gamma_{21}\left(\gamma_{13}\left(r_2-v_2\right)-\gamma_{31}r_2\right)^2+\gamma_{31}^2r_2^2\right)}{\gamma_{31}r_2\left(1-p\right)}<r_2-v_2
\]
The third eigenvalue is negative when:
\[
r_2-v_2>0
\]
Therefore we can say that the $xz$-boundary equilibrium is stable if $0<r_2-v_2$, $\displaystyle r_2-v_2<\frac{\gamma_{31}r_2}{\gamma_{13}}$, and:
\[
\frac{r_1v_1\left(\gamma_{21}\left(\gamma_{13}\left(r_2-v_2\right)-\gamma_{31}r_2\right)^2+\gamma_{31}^2r_2^2\right)}{\gamma_{31}r_2\left(1-p\right)}<r_2-v_2
\]

\subsection{Analyzing the $yz$-boundary equilibrium}\label{subsec:stability_yz_boundary_equilibrium}
Plugging the $yz$-boundary equilibrium into \myref[Matrix]{eq:5.1} yields:
\begin{equation}
    \textbf{J}=\begin{bmatrix}
        j_{11} & 0 & 0\\
        0 & j_{22} & j_{23}\\
        0 & j_{32} & j_{33}
    \end{bmatrix}
    \label{eq:5.14}
\end{equation}
where
\begin{align*}
    j_{11} &= 1+\gamma_{12}\left(\bar{y}\right)^2-\gamma_{13}\bar{z}\\
    j_{22} &= r_1\left(1-2\bar{y}\right)-\frac{v_1\left(1-p\right)\bar{z}}{\left(v_1+\left(1-p\right)\bar{y}\right)^2}\\
    j_{23} &= -\frac{\left(1-p\right)\bar{y}}{v_1+\left(1-p\right)\bar{y}}\\
    j_{32} &= \frac{v_1v_3\left(1-p\right)\bar{z}}{\left(v_1+\left(1-p\right)\bar{y}\right)^2}\\
    j_{33} &= r_2\left(1-2\gamma_{31}\bar{z}\right)+\frac{v_3\left(1-p\right)\bar{y}}{v_1+\left(1-p\right)\bar{y}}-v_2
\end{align*}
The characteristic equation for this Jacobian matrix is:
\begin{equation}
    -\left(j_{11}-\lambda\right)\left(\lambda^2-\left(j_{22}+j_{33}\right)\lambda+j_{22}j_{33}-j_{23}j_{32}\right)=0
    \label{eq:5.15}
\end{equation}
Solving for the eigenvalues in \myref[Equation]{eq:5.15}, we get:
\[
\lambda=\left\{
j_{11},\ 
\frac{\left(j_{22}+j_{33}\right)\pm\sqrt{\left(j_{22}-j_{33}\right)^2+4j_{23}j_{32}}}{2}
\right\}
\]
The first eigenvalue is negative when:
\[
\frac{1+\gamma_{12}\bar{y}}{\gamma_{13}}<\bar{z}
\]
The other two eigenvalues are negative when:
\[
j_{23}j_{32}<j_{22}j_{33}
\]
Therefore we can say that the $xy$-boundary equilibrium is stable if:
\[
\frac{1+\gamma_{12}\bar{y}}{\gamma_{13}}<\bar{z},\quad 
j_{23}j_{32}<j_{22}j_{33}
\]
where
\begin{align*}
    j_{22} &= -\frac{v_1\left(1-p\right)\bar{z}}{\left(v_1+\left(1-p\right)\bar{y}\right)^2}+r_1\left(1-2\bar{y}\right)\\
    j_{23} &= -\frac{\left(1-p\right)\bar{y}}{v_1+\left(1-p\right)\bar{y}}\\
    j_{32} &= \frac{v_1v_3\left(1-p\right)\bar{z}}{\left(v_1+\left(1-p\right)\bar{y}\right)^2}\\
    j_{33} &= r_2\left(1-2\gamma_{31}\bar{z}\right)+\frac{v_3\left(1-p\right)\bar{y}}{v_1+\left(1-p\right)\bar{y}}-v_2
\end{align*}

\subsection{Analyzing the interior equilibrium}\label{subsec:stability_interior_equilibrium}
Plugging the interior equilibrium into \myref[Matrix]{eq:5.1} yields:
\begin{equation}
    \textbf{J}=\begin{bmatrix}
        j_{11} & j_{12} & j_{13}\\
        j_{21} & j_{22} & j_{23}\\
        0 & j_{32} & j_{33}
    \end{bmatrix}
    \label{eq:5.16}
\end{equation}
where
\begin{align*}
    j_{11} &= 1-2x^*+\gamma_{12}\left(y^*\right)^2-\gamma_{13}z^*\\
    j_{12} &= 2\gamma_{12}x^*y^*\\
    j_{13} &= -\gamma_{13}x^*\\
    j_{21} &= 2\gamma_{21}r_1x^*y^*\\
    j_{22} &= r_1\left(1-2y^*+\gamma_{21}\left(x^*\right)^2\right)-\frac{v_1\left(1-p\right)z^*}{\left(v_1+\left(1-p\right)y^*\right)^2}\\
    j_{23} &= -\frac{\left(1-p\right)y^*}{v_1+\left(1-p\right)y^*}\\
    j_{32} &= \frac{v_1v_3\left(1-p\right)z^*}{\left(v_1+\left(1-p\right)y^*\right)^2}\\
    j_{33} &= r_2\left(1-2\gamma_{31}z^*\right)+\frac{v_3\left(1-p\right)y^*}{v_1+\left(1-p\right)y^*}-v_2
\end{align*}
The characteristic equation for this Jacobian matrix is:
\begin{equation}
    \lambda^3+J_2\lambda^2+J_1\lambda+J_0=0
    \label{eq:5.17}
\end{equation}
where
\begin{align*}
    J_2 &= -\left(j_{11}+j_{22}+j_{33}\right)\\
    J_1 &= j_{11}\left(j_{22}+j_{33}\right)+j_{22}j_{33}-\left(j_{12}j_{21}+j_{23}j_{32}\right)\\
    J_0 &= -j_{13}j_{21}j_{32}+j_{12}j_{21}j_{33}+j_{11}j_{23}j_{32}-j_{11}j_{22}j_{33}
\end{align*}
By the Routh–Hurwitz stability criterion~\cite{routh1877treatise}, we can say that the interior equilibrium is stable if:
\[
J_2>0,\quad J_1>0,\quad J_0>0,\quad J_2J_1>J_0
\]
where
\begin{align*}
    J_2 &= -\left(j_{11}+j_{22}+j_{33}\right)\\
    J_1 &= j_{11}\left(j_{22}+j_{33}\right)+j_{22}j_{33}-\left(j_{12}j_{21}+j_{23}j_{32}\right)\\
    J_0 &= -j_{13}j_{21}j_{32}+j_{12}j_{21}j_{33}+j_{11}j_{23}j_{32}-j_{11}j_{22}j_{33}\\
    j_{11} &= 1-2x^*+\gamma_{12}\left(y^*\right)^2-\gamma_{13}z^*\\
    j_{12} &= 2\gamma_{12}x^*y^*\\
    j_{13} &= -\gamma_{13}x^*\\
    j_{21} &= 2\gamma_{21}r_1x^*y^*\\
    j_{22} &= r_1\left(1-2y^*+\gamma_{21}\left(x^*\right)^2\right)-\frac{v_1\left(1-p\right)z^*}{\left(v_1+\left(1-p\right)y^*\right)^2}\\
    j_{23} &= -\frac{\left(1-p\right)y^*}{v_1+\left(1-p\right)y^*}\\
    j_{32} &= \frac{v_1v_3\left(1-p\right)z^*}{\left(v_1+\left(1-p\right)y^*\right)^2}\\
    j_{33} &= r_2\left(1-2\gamma_{31}z^*\right)+\frac{v_3\left(1-p\right)y^*}{v_1+\left(1-p\right)y^*}-v_2
\end{align*}
    % ----------------------------------------------------------------------------
% Author: Rayla Kurosaki
% GitHub: https://github.com/rkp1503
% ----------------------------------------------------------------------------

\section{Numerical Simulations}\label{sec:numerical_simulations}
In \myref[Section]{sec:equilibria-analysis}, we used mathematical analysis to compute the equilibria that exists in \myref[Model]{model:rayla-ephraim} and determined the conditions for stability of each one. In this section, we will support and verify the stable equilibria determined in \myref[Section]{sec:equilibria-analysis} through numerical simulations. Also, we will show the existence of a hopf bifurcation for the interior equilibrium.

\begin{figure}[H]
    \centering
    \subfloat[$z$-axial equilibrium with the set of parameters~\eqref{params:axial-z}]{%
    \resizebox*{7cm}{!}{\includegraphics{equilibrium-axial-z.eps}\label{fig:axial-z}}}\hspace{5pt}
    \subfloat[$xy$-boundary equilibrium with the set of parameters~\eqref{params:boundary-xy}]{%
    \resizebox*{7cm}{!}{\includegraphics{equilibrium-boundary-xy.eps}\label{fig:boundary-xy}}}\hspace{5pt}
    \subfloat[$xz$-boundary equilibrium with the set of parameters~\eqref{params:boundary-xz}]{%
    \resizebox*{7cm}{!}{\includegraphics{equilibrium-boundary-xz.eps}\label{fig:boundary-xz}}}\hspace{5pt}
    \subfloat[$yz$-boundary equilibrium with the set of parameters~\eqref{params:boundary-yz}]{%
    \resizebox*{7cm}{!}{\includegraphics{equilibrium-boundary-yz.eps}\label{fig:boundary-yz}}}
    \caption{Showing the stability of non-interior equilibria for different set of parameters.}
    \label{fig:semi-trivial-equilibria-plots}
\end{figure}

\subsection{The $z$-axial equilibrium}\label{subsec:numsim_z_axial_equilibrium}
By \myref[Theorem]{thm:axial-z-exist} and \myref[Theorem]{thm:axial-z-stability}, we know that the $z$-axial equilibrium
\begin{equation*}
    E_z=\left(0,0,\frac{r_2-v_2}{\gamma_{31}r_2}\right)
\end{equation*}
exists if the condition $r_{zx} > u_4$ is satisfied and is stable if the following conditions are satisfied:
\begin{equation*}
    \frac{u_4}{r_{zx}} < 1-\frac{1}{\varphi_{xz}},\quad
    \frac{u_4}{r_{zx}} < 1-\frac{r_{yx}u_2}{u_1\left(1-p\right)},\quad
    \frac{u_4}{r_{zx}} < \frac{1}{2}
\end{equation*}
To satisfy the conditions above, lets consider the following set of parameters:
\begin{equation}\label{params:axial-z}
    \begin{dcases}
        \begin{aligned}
            r_{yx} &= 0.007\\
            r_{zx} &= 1.136\\
            p &= 0.874
        \end{aligned}
    \end{dcases},\quad 
    \begin{dcases}
        \begin{aligned}
            \varphi_{xy} &= 0.318\\
            \varphi_{yx} &= 0.416\\
            \varphi_{xz} &= 1.59
        \end{aligned}
    \end{dcases},\quad 
    \begin{dcases}
        \begin{aligned}
            u_1 &= 1.655\\
            u_2 &= 0.791\\
            u_3 &= 0.994\\
            u_4 &= 0.356
        \end{aligned}
    \end{dcases}
\end{equation}
Under this set of parameter values, the $z$-axial equilibrium is $E_z=(0,0,0.6866)$. This is further supported by \myref[Figure]{fig:axial-z}, which is the result of numerically solving \myref[Model]{model:rayla-ephraim}.

\subsection{The $xy$-boundary equilibrium}\label{subsec:numsim_xy_boundary_equilibrium}
By \myref[Theorem]{thm:boundary-xy-exist} and \myref[Theorem]{thm:boundary-xy-stability}, we know that the $xy$-boundary equilibrium $E_{xy}=\left(x^*,y^*,0\right)$ exists where $x^*=1+\varphi_{xy}\left(y^*\right)^2$ and $y^*$ is a positive solution to the equation:
\begin{equation*}
    \varphi_{xy}^2\varphi_{yx}\left(y^*\right)^4+2\varphi_{xy}\varphi_{yx}\left(y^*\right)^2-y^*+\varphi_{yx}+1=0
\end{equation*}
which can be achieved under the following condition
\begin{equation*}
    \varphi_{yx}<\frac{\beta-1}{\left(\varphi_{xy}\beta^2+1\right)^2}
\end{equation*}
for some $\beta\in\left(1, \infty\right)$ and the equilibrium is locally stable when $C_2>0,\ C_1>0,\ C_0>0,\ C_2C_1>C_0$ where:
\begin{align*}
    C_2 &= -j_{11}-j_{22}-j_{33}\\
    C_1 &= j_{11}j_{22}+j_{11}j_{33}+j_{22}j_{33}-j_{12}j_{21}\\
    C_0 &= j_{33}\left(j_{12}j_{21}-j_{11}j_{22}\right)\\
    j_{11} &= 1-2x+\varphi_{xy}\left(y^*\right)^2\\
    j_{12} &= 2\varphi_{xy}x^*y^*\\
    j_{21} &= 2r_{yx}\varphi_{yx}x^*y^*\\
    j_{22} &= r_{yx}\left(1-2y^*+\varphi_{yx}\left(x^*\right)^2\right)\\
    j_{33} &= r_{zx}+\frac{u_3\left(1-p\right)y^*}{u_2+\left(1-p\right)y^*}-u_4
\end{align*}
To satisfy the conditions above, we will let $\beta=11$ and consider the following set of parameters:
\begin{equation}\label{params:boundary-xy}
    \begin{dcases}
        \begin{aligned}
            r_{yx} &= 0.049\\
            r_{zx} &= 0.467\\
            p &= 0.645
        \end{aligned}
    \end{dcases},\quad 
    \begin{dcases}
        \begin{aligned}
            \varphi_{xy} &= 0.024\\
            \varphi_{yx} &= 0.163\\
            \varphi_{xz} &= 0.031
        \end{aligned}
    \end{dcases},\quad
    \begin{dcases}
        \begin{aligned}
            u_1 &= 0.31\\
            u_2 &= 0.978\\
            u_3 &= 0.9\\
            u_4 &= 1.004
        \end{aligned}
    \end{dcases}
\end{equation}
Under this set of parameter values, the $xy$-boundary equilibrium is $E_{xy}=(1.0331,1.174,0)$. This is further supported by \myref[Figure]{fig:boundary-xy}, which is the result of numerically solving \myref[Model]{model:rayla-ephraim}.

\subsection{The $xz$-boundary equilibrium}\label{subsec:numsim_xz_boundary_equilibrium}
By \myref[Theorem]{thm:boundary-xz-exist} and \myref[Theorem]{thm:boundary-xz-stability}, we know that the $xz$-boundary equilibrium $E_{xz}=\left(x^*,\ 0,\ z^*\right)$ exist where
\begin{equation*}
    x^*=1-\varphi_{xz}\left(1-\frac{u_4}{r_{zx}}\right),\quad
    z^*=1-\frac{u_4}{r_{zx}}
\end{equation*}
provided that the conditions have been satisfied.
\begin{equation*}
    \frac{u_4}{r_{zx}}+\frac{1}{\varphi_{xz}} > 1,\quad 
    r_{zx}>u_4
\end{equation*}
and the equilibrium is locally stable when:
\begin{equation*}
    z^*>\frac{1-2x^*}{\varphi_{xz}},\quad
    z^*>\frac{r_{yx}u_2\left(1+\varphi_{yx}\left(x^*\right)^2\right)}{u_1\left(1-p\right)},\quad
    z^*>\frac{r_{zx}-u_4}{2r_{zx}}
\end{equation*}
To satisfy the conditions above, lets consider the following set of parameters:
\begin{equation}\label{params:boundary-xz}
    \begin{dcases}
        \begin{aligned}
            r_{yx} &= 0.199\\
            r_{zx} &= 1.494\\
            p &= 0.482
        \end{aligned}
    \end{dcases},\quad 
    \begin{dcases}
        \begin{aligned}
            \varphi_{xy} &= 0.449\\
            \varphi_{yx} &= 0.993\\
            \varphi_{xz} &= 1.152
        \end{aligned}
    \end{dcases},\quad
    \begin{dcases}
        \begin{aligned}
            u_1 &= 1.671\\
            u_2 &= 0.663\\
            u_3 &= 1.556\\
            u_4 &= 1.04
        \end{aligned}
    \end{dcases}
\end{equation}
Under this set of parameter values, the $xz$-boundary equilibrium is $E_{xz}=(0.6499,0,0.3039)$. This is further supported by \myref[Figure]{fig:boundary-xz}, which is the result of numerically solving \myref[Model]{model:rayla-ephraim}.

\subsection{The $yz$-boundary equilibrium}\label{subsec:numsim_yz_boundary_equilibrium}
By \myref[Theorem]{thm:boundary-yz-exist} and \myref[Theorem]{thm:boundary-yz-stability}, we know that the $yz$-boundary equilibrium $E_{yz}=\left(0,\ y^*,\ z^*\right)$ exists where
\begin{equation*}
    z^*=1+\frac{1}{r_{zx}}\left(\frac{u_3\left(1-p\right)y^*}{u_2+\left(1-p\right)y^*}-u_4\right)
\end{equation*}
and $y^*$ is a positive solution to 
\begin{equation*}
    \frac{Y_3\left(y^*\right)^3+Y_2\left(y^*\right)^2+Y_1y^*+Y_0}{r_{zx}\left(u_2+\left(1-p\right)y^*\right)^2}=0
\end{equation*}
where:
\begin{align*}
    Y_3 &= -r_{yx}r_{zx}\left(1-p\right)^2\\
    Y_2 &= r_{yx}r_{zx}\left(1-p\right)\left(\left(1-p\right)-2u_2\right)\\
    Y_1 &= u_1\left(u_4-u_3-r_{zx}\right)\left(1-p\right)^2+r_{yx}r_{zx}u_2\left(2\left(1-p\right)-u_2\right)\\
    Y_0 &= u_2\left(r_{yx}r_{zx}u_2+u_1\left(u_4-r_2\right)\left(1-p\right)\right)
\end{align*}
provided that the following conditions are satisfied:
\begin{equation*}
    y^* > \frac{u_2\left(u_4-r_{zx}\right)}{\left(u_3-u_4+r_{zx}\right)\left(1-p\right)},\quad 
    1 > \frac{u_1\left(r_2-u_4\right)\left(1-p\right)}{r_{yx}r_{zx}u_2}
\end{equation*}
and the equilibrium is locally stable when $C_2>0,\ C_1>0,\ C_0>0,\ C_2C_1>C_0$ where:
\begin{align*}
    C_2 &= -j_{11}-j_{22}-j_{33}\\
    C_1 &= j_{11}j_{22}+j_{11}j_{33}+j_{22}j_{33}-j_{23}j_{32}\\
    C_0 &= j_{11}\left(j_{23}j_{32}-j_{22}j_{33}\right)\\
    j_{11} &= 1+\varphi_{xy}\left(y^*\right)^2-\varphi_{xz}z^*\\
    j_{22} &= r_{yx}\left(1-2y^*\right)-\frac{u_1u_2\left(1-p\right)z^*}{\left(u_2+\left(1-p\right)y^*\right)^2}\\
    j_{23} &= -\frac{u_1\left(1-p\right)y^*}{u_2+\left(1-p\right)y^*}\\
    j_{32} &= \frac{u_2u_3\left(1-p\right)z^*}{\left(u_2+\left(1-p\right)y\right)^2}\\
    j_{33} &= r_{zx}\left(1-2z^*\right)+\frac{u_3\left(1-p\right)y^*}{u_2+\left(1-p\right)y^*}-u_4
\end{align*}
To satisfy the conditions above, lets consider the following set of parameters:
\begin{equation}\label{params:boundary-yz}
    \begin{dcases}
        \begin{aligned}
            r_{yx} &= 1.219\\
            r_{zx} &= 0.452\\
            p &= 0.589
        \end{aligned}
    \end{dcases},\quad 
    \begin{dcases}
        \begin{aligned}
            \varphi_{xy} &= 0.047\\
            \varphi_{yx} &= 1.587\\
            \varphi_{xz} &= 1.908
        \end{aligned}
    \end{dcases},\quad
    \begin{dcases}
        \begin{aligned}
            u_1 &= 1.658\\
            u_2 &= 1.812\\
            u_3 &= 1.473\\
            u_4 &= 0.289
        \end{aligned}
    \end{dcases}
\end{equation}
Under this set of parameter values, the $yz$-boundary equilibrium is $E_{yz}=(0,0.7773,0.8491)$. This is further supported by \myref[Figure]{fig:boundary-yz}, which is the result of numerically solving \myref[Model]{model:rayla-ephraim}.

\subsection{The interior equilibrium}\label{subsec:numsim_interior_equilibrium}
By \myref[Theorem]{thm:boundary-yz-exist} and \myref[Theorem]{thm:boundary-yz-stability}, we know that the interior equilibrium $E_{xyz}=\left(x^*,\ y^*,\ z^*\right)$ exists where
\begin{equation*}
    x^*=1+\varphi_{xy}\left(y^*\right)^2-\varphi_{xz}z^*,\quad 
    z^*=1+\frac{1}{r_{zx}}\left(\frac{u_3\left(1-p\right)y^*}{u_2+\left(1-p\right)y^*}-u_4\right)
\end{equation*}
and $y^*$ is a positive solution to 
\begin{equation*}
    \frac{Y_6\left(y^*\right)^6+Y_5\left(y^*\right)^5+Y_4\left(y^*\right)^4+Y_3\left(y^*\right)^3+Y_2\left(y^*\right)^2+Y_1y^*+Y_0}{r_{zx}^2\left(u_2+\left(1-p\right)y^*\right)^3}=0
\end{equation*}
where:
\begin{align*}
    Y_6 &= r_{yx}r_{zx}^2\varphi_{xy}^2\varphi_{yx}\left(1-p\right)^2\\
    Y_5 &= 2r_{yx}r_{zx}^2u_2\varphi_{xy}^2\varphi_{yx}\left(1-p\right)\\
    Y_4 &= r_{yx}r_{zx}\varphi_{xy}\varphi_{yx}\left(2\left(r_{zx}\left(1-\varphi_{xz}\right)+\varphi_{xz}\left(u_4-u_3\right)\right)\left(1-p\right)^2+r_{zx}u_2^2\varphi_{xy}\right)\\
    Y_3 &= r_{yx}r_{zx}\left(1-p\right)\left(-r_{zx}\left(1-p\right)+2u_2\varphi_{xy}\varphi_{yx}\left(2r_{zx}\left(1-\varphi_{xz}\right)+\varphi_{xz}\left(2u_4-u_3\right)\right)\right)\\
    Y_2 &= r_{yx}\left(\left(\varphi_{yx}\left(r_{zx}\left(1-\varphi_{xz}\right)+\varphi_{xz}\left(u_4-u_3\right)\right)^2+r_{zx}^2\right)\left(1-p\right)^2-2r_{zx}^2u_2\left(1-p\right)\right.\\
    &\left.+2r_{zx}\varphi_{xy}\varphi_{yx}u_2^2\left(r_{zx}\left(1-\varphi_{xz}\right)+u_4\varphi_{xz}\right)\right)\\
    Y_1 &= r_{zx}u_1\left(u_4-u_3-r_{zx}\right)\left(1-p\right)^2+2r_{yx}u_2\left(r_{zx}^2\left(\varphi_{yx}\left(\varphi_{xz}-1\right)^2+1\right)\right.\\
    &\left.+\varphi_{xz}\varphi_{yx}\left(-u_4\left(2r_{zx}\left(\varphi_{xz}-1\right)+\varphi_{xz}u_3\right)+r_{zx}u_3\left(\varphi_{xz}-1\right)+\varphi_{xz}u_4^2\right)\right)\left(1-p\right)\\
    &-r_{yx}r_{zx}^2u_2^2\\
    Y_0 &= u_2\left(r_{zx}u_1\left(u_4-r_{zx}\right)\left(1-p\right)+r_{yx}u_2\left(\varphi_{yx}\left(\varphi_{xz}\left(r_{zx}-u_4\right)-r_{zx}\right)^2+r_{zx}^2\right)\right)
\end{align*}
provided that the following conditions are satisfied:
\begin{equation*}
    \frac{1+\varphi_{xy}\left(y^*\right)^2}{\varphi_{xz}}>z^*,\quad
    y^*>\frac{u_2\left(u_4-r_{zx}\right)}{\left(u_3-\left(u_4-r_{zx}\right)\right)\left(1-p\right)},\quad
    Y_0 < 0
\end{equation*}
and the equilibrium is locally stable when $C_2>0,\ C_1>0,\ C_0>0,\ C_2C_1>C_0$ where:
\begin{align*}
    C_2 &= -j_{11}-j_{22}-j_{33}\\
    C_1 &= j_{11}j_{22}+j_{11}j_{33}+j_{22}j_{33}-j_{12}j_{21}-j_{23}j_{32}\\
    C_0 &= j_{11}\left(j_{23}j_{32}-j_{22}j_{33}\right)+j_{21}\left(j_{12}j_{33}-j_{13}j_{32}\right)\\
    j_{11} &= 1-2x^*+\varphi_{xy}\left(y^*\right)^2-\varphi_{xz}z^*\\
    j_{12} &= 2\varphi_{xy}x^*y^*\\
    j_{13} &= -\varphi_{xz}x^*\\
    j_{21} &= 2r_{yx}\varphi_{yx}x^*y^*\\
    j_{22} &= r_{yx}\left(1-2y^*+\varphi_{yx}\left(x^*\right)^2\right)-\frac{u_1u_2\left(1-p\right)z^*}{\left(u_2+\left(1-p\right)y^*\right)^2}\\
    j_{23} &= -\frac{u_1\left(1-p\right)y^*}{u_2+\left(1-p\right)y^*}\\
    j_{32} &= \frac{u_2u_3\left(1-p\right)z^*}{\left(u_2+\left(1-p\right)y^*\right)^2}\\
    j_{33} &= r_{zx}\left(1-2z^*\right)+\frac{u_3\left(1-p\right)y^*}{u_2+\left(1-p\right)y^*}-u_4
\end{align*}
To ensure that the interior equilibrium exist and is stable, lets consider the following set of parameters:
\begin{equation}\label{params:interior-a}
    \begin{dcases}
        \begin{aligned}
            r_{yx} &= 0.5\\
            r_{zx} &= 0.5\\
            p &= 0.6
        \end{aligned}
    \end{dcases},\quad 
    \begin{dcases}
        \begin{aligned}
            \varphi_{xy} &= 0.6\\
            \varphi_{yx} &= 0.15\\
            \varphi_{xz} &= 0.4
        \end{aligned}
    \end{dcases},\quad
    \begin{dcases}
        \begin{aligned}
            u_1 &= 0.6\\
            u_2 &= 0.08\\
            u_3 &= 0.5\\
            u_4 &= 0.5
        \end{aligned}
    \end{dcases}
\end{equation}
Under this set of parameter values, the interior equilibrium is $E_{xyz}=(0.9099,0.0599,0.2305)$. This is further supported by the four figures in \myref[Figure]{fig:nontrivial-equilibria-plots} where \myref[Figure]{fig:time-evolution} shows the time evolution of each Species, \myref[Figure]{fig:phase-plane-3d} shows the phase portrait, and \myref[Figure]{fig:phase-plane-xy}, \myref[Figure]{fig:phase-plane-xz}, and \myref[Figure]{fig:phase-plane-yz} are phase planes when numerically solving \myref[Model]{model:rayla-ephraim}.

For \myref[Model]{model:rayla-ephraim}, we can numerically show that a hopf bifurcation exists for each parameter. Starting with $r_{zx}$, we will plot the time evolution of the ecosystem at $r_{zx}=0.35$ to show that the ecosystem expresses an oscillatory behavior as shown in \myref[Figure]{fig:bifurcation-r_zx-xyz}. Then, we will generate a bifurcation diagram for Species $X,\ Y,\ Z$ over a set interval of $r_{zx}$, expressed in \myref[Figure]{fig:bifurcation-r_zx-x}, \myref[Figure]{fig:bifurcation-r_zx-y}, \myref[Figure]{fig:bifurcation-r_zx-z} respectively. For $r_{zx}$, the interval is $r_{zx}\in(0.133,0.6155)$. From the bifurcation diagrams, we can see that the ecosystem undergoes 2 changes. Denoting the stable solutions for Species $X,\ Y,\ Z$ in black, red, and blue respectively and denoting the unstable solutions in green, we can see that the ecosystem starts off in a stable state and then becomes unstable when $r_{zx}\approx 0.29$. From here, this behavior is maintained until $r_{zx}\approx 0.47$ where it transitions back to a stable state. Thus, we can say that for the set of \myref[parameters]{params:interior-a}, the ecosystem maintains a stable equilibrium when $r_{zx}\in(0.29,0.47)$ and displays an oscillatory behavior when $r_{zx}\in(0.133,0.29)$ and $r_{zx}\in(0.47,0.6155)$.
% If $r_{zx}>0.6155$, then Species $y$ dies out, turning the interior equilibrium to a $xz$-boundary equilibrium. If $r_{zx}<0.133$, then....

We can repeat this process using the same set of parameters to show that a hopf bifurcation exists for $p$, $\varphi_{yx}$, and $u_2$. For $p$, the ecosystem undergoes a hopf bifurcation at $p\approx 0.371$, shown in \myref[Figure]{fig:bifurcation-p}. For $\varphi_{yx}$, the ecosystem undergoes a hopf bifurcation at $p\approx 0.387$, shown in \myref[Figure]{fig:bifurcation-phi_yx}. For $u_2$, the ecosystem undergoes a hopf bifurcation at $u_2\approx 0.051$, shown in \myref[Figure]{fig:bifurcation-u_2}. For the other parameters $r_{yx},\ \varphi_{xy},\ \varphi_{xz},\ u_1,\ u_3,\ u_4$, we will consider the set of \myref[parameters]{params:interior-b}. Applying the above procedure to these parameters, we can conclude that the ecosystem undergoes a hopf bifurcation at $r_{yx}\approx 0.66,\ \varphi_{xy}\approx 0.125,\ \varphi_{xz}\approx\{0.402,1.342\},\ u_1\approx 0.728,\ u_3\approx\{0.511,2.501\},\ u_4\approx\{0.122,0.314\}$, depicted in \myref[Figure]{fig:bifurcation-r_yx}, \myref[Figure]{fig:bifurcation-phi_xz}, \myref[Figure]{fig:bifurcation-u_1}, \myref[Figure]{fig:bifurcation-u_3}, \myref[Figure]{fig:bifurcation-u_4} respectively.
\begin{equation}\label{params:interior-b}
    \begin{dcases}
        \begin{aligned}
            r_{yx} &= 0.5\\
            r_{zx} &= 0.5\\
            p &= 0.6
        \end{aligned}
    \end{dcases},\quad 
    \begin{dcases}
        \begin{aligned}
            \varphi_{xy} &= 0.6\\
            \varphi_{yx} &= 0.15\\
            \varphi_{xz} &= 0.4
        \end{aligned}
    \end{dcases},\quad
    \begin{dcases}
        \begin{aligned}
            u_1 &= 0.6\\
            u_2 &= 0.08\\
            u_3 &= 0.5\\
            u_4 &= 0.5
        \end{aligned}
    \end{dcases}
\end{equation}


    %-------------------------------------------------------------------------
    % Bibliography
    %-------------------------------------------------------------------------
    \newpage
    % \nocite{*}
    \printbibliography[title={Bibliography}]

    %-------------------------------------------------------------------------
    % Appendix
    %-------------------------------------------------------------------------
    \newpage
    \appendix
    % ----------------------------------------------------------------------------
% Author: Ramsey (Rayla) Phuc
% Alias: Rayla Kurosaki
% GitHub: https://github.com/rkp1503
% 
% Co-author: Ephraim Agyingi
% ----------------------------------------------------------------------------
\chapter{Figures}\label{chapter:figures}
\begin{figure}[H]
    \centering
    \begin{subfigure}[b]{0.47\textwidth}
        \centering
        \includegraphics[width=\textwidth]{z_axial.png}
        \caption{$z$-axial equilibria; $r_1 = 0.404$, $r_2 = 0.903$, $p = 0.182$, $\gamma_{12} = 0.639$, $\gamma_{21} = 0.283$, $\gamma_{13} = 0.301$, $\gamma_{31} = 0.110$, $v_1 = 0.645$, $v_2 = 0.175$, $v_3 = 0.145$.}
        \label{fig:z_axial}
    \end{subfigure}
    \hfill
    \begin{subfigure}[b]{0.47\textwidth}
        \centering
        \includegraphics[width=\textwidth]{xy_boundary.png}
        \caption{$xy$-boundary equilibria; $r_1 = 0.978$, $r_2 = 0.613$, $p = 0.326$, $\gamma_{12} = 0.245$, $\gamma_{21} = 0.015$, $\gamma_{13} = 0.920$, $\gamma_{31} = 0.696$, $v_1 = 0.523$, $v_2 = 0.951$, $v_3 = 0.570$.}
        \label{fig:xy_boundary}
    \end{subfigure}
    \hfill
    \begin{subfigure}[b]{0.47\textwidth}
        \centering
        \includegraphics[width=\textwidth]{xz_boundary.png}
        \caption{$xz$-boundary equilibria; $r_1 = 0.102$, $r_2 = 0.763$, $p = 0.271$, $\gamma_{12} = 0.182$, $\gamma_{21} = 0.301$, $\gamma_{13} = 0.109$, $\gamma_{31} = 0.198$, $v_1 = 0.983$, $v_2 = 0.186$, $v_3 = 0.113$.}
        \label{fig:xz_boundary}
    \end{subfigure}
    \hfill
    \begin{subfigure}[b]{0.47\textwidth}
        \centering
        \includegraphics[width=\textwidth]{yz_boundary.png}
        \caption{$yz$-boundary equilibria; $r_1 = 0.978$, $r_2 = 0.310$, $p = 0.843$, $\gamma_{12} = 0.002$, $\gamma_{21} = 0.407$, $\gamma_{13} = 0.859$, $\gamma_{31} = 0.446$, $v_1 = 0.872$, $v_2 = 0.201$, $v_3 = 0.959$.}
        \label{fig:yz_boundary}
    \end{subfigure}
       \caption{Showing the stability of equilibrium points with different set of parameters.}
       \label{fig:semi-trivial-equilibria-plots}
\end{figure}

\begin{figure}[H]
    \centering
    \begin{subfigure}[b]{0.49\textwidth}
        \centering
        \includegraphics[width=\textwidth]{interior.png}
        \caption{Stability of the interior equilibrium.}
        \label{fig:interior}
    \end{subfigure}
    \hfill
    \begin{subfigure}[b]{0.49\textwidth}
        \centering
        \includegraphics[width=\textwidth]{interior_pp_3D.png}
        \caption{3D phase portrait.}
        \label{fig:phase_plane_3d}
    \end{subfigure}
    \hfill
    \begin{subfigure}[b]{0.49\textwidth}
        \centering
        \includegraphics[width=\textwidth]{interior_pp_xy.png}
        \caption{$xy$ phase plane.}
        \label{fig:phase_plane_xy}
    \end{subfigure}
    \hfill
    \begin{subfigure}[b]{0.49\textwidth}
        \centering
        \includegraphics[width=\textwidth]{interior_pp_xz.png}
        \caption{$xz$ phase plane.}
        \label{fig:phase_plane_xz}
    \end{subfigure}
    \hfill
    \begin{subfigure}[b]{0.49\textwidth}
        \centering
        \includegraphics[width=\textwidth]{interior_pp_yz.png}
        \caption{$yz$ phase plane.}
        \label{fig:phase_plane_yz}
    \end{subfigure}
       \caption{Different types of plots to show the behavior of \myref[Model]{model:3.2} where $r_1 = 0.635$, $r_2 = 0.742$, $p = 0.853$, $\gamma_{12} = 0.142$, $\gamma_{21} = 0.002$, $\gamma_{13} = 0.148$, $\gamma_{31} = 0.215$, $v_1 = 0.090$, $v_2 = 0.891$, $v_3 = 0.980$.}
       \label{fig:nontrivial-equilibria-plots}
\end{figure}


\end{document}
